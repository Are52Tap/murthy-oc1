\documentstyle[jair,twoside,11pt,theapa]{article}
% Psfig/TeX 
\def\PsfigVersion{1.10}
\def\setDriver{\DvipsDriver} % \DvipsDriver or \OzTeXDriver
%
% All software, documentation, and related files in this distribution of
% psfig/tex are Copyright 1993 Trevor J. Darrell
%
% Permission is granted for use and non-profit distribution of psfig/tex 
% providing that this notice is clearly maintained. The right to
% distribute any portion of psfig/tex for profit or as part of any commercial
% product is specifically reserved for the author(s) of that portion.
%
% To use with LaTeX, use \documentstyle[psfig,...]{...}
% To use with TeX, use \input psfig.sty
%
% Bugs and improvements to trevor@media.mit.edu.
%
% Thanks to Ned Batchelder, Greg Hager (GDH), J. Daniel Smith (JDS),
% Tom Rokicki (TR), Robert Russell (RR), George V. Reilly (GVR),
% Ken McGlothlen (KHC), Baron Grey (BG), Gerhard Tobermann (GT).
% and all others who have contributed code and comments to this project!
%
% ======================================================================
% Modification History:
%
%  9 Oct 1990   JDS	used more robust bbox reading code from Tom Rokicki
% 29 Mar 1991   JDS	implemented rotation= option
% 25 Jun 1991   RR	if bb specified on cmd line don't check
%			for .ps file.
%  3 Jul 1991	JDS	check if file already read in once
%  4 Sep 1991	JDS	fixed incorrect computation of rotated
%			bounding box
% 25 Sep 1991	GVR	expanded synopsis of \psfig
% 14 Oct 1991	JDS	\fbox code from LaTeX so \psdraft works with TeX
%			changed \typeout to \ps@typeout
% 17 Oct 1991	JDS	added \psscalefirst and \psrotatefirst
% 23 Jun 1993   KHC     ``doclip'' must appear before ``rotate''
% 27 Oct 1993   TJD	removed printing of filename to avoid 
%			underscore problems. changed \frame to \fbox.
%			Added OzTeX support from BG. Added new
%			figure search path code from GT.
%
% ======================================================================
%
% Command synopsis:
%
% \psdraft	draws an outline box, but doesn't include the figure
%		in the DVI file.  Useful for previewing.
%
% \psfull	includes the figure in the DVI file (default).
%
% \psscalefirst width= or height= specifies the size of the figure
% 		before rotation.
% \psrotatefirst (default) width= or height= specifies the size of the
% 		 figure after rotation.  Asymetric figures will
% 		 appear to shrink.
%
% \psfigurepath{dir:dir:...}  sets the path to search for the figure
%
% \psfig
% usage: \psfig{file=, figure=, height=, width=,
%			bbllx=, bblly=, bburx=, bbury=,
%			rheight=, rwidth=, clip=, angle=, silent=}
%
%	"file" is the filename.  If no path name is specified and the
%		file is not found in the current directory,
%		it will be looked for in directory \psfigurepath.
%	"figure" is a synonym for "file".
%	By default, the width and height of the figure are taken from
%		the BoundingBox of the figure.
%	If "width" is specified, the figure is scaled so that it has
%		the specified width.  Its height changes proportionately.
%	If "height" is specified, the figure is scaled so that it has
%		the specified height.  Its width changes proportionately.
%	If both "width" and "height" are specified, the figure is scaled
%		anamorphically.
%	"bbllx", "bblly", "bburx", and "bbury" control the PostScript
%		BoundingBox.  If these four values are specified
%               *before* the "file" option, the PSFIG will not try to
%               open the PostScript file.
%	"rheight" and "rwidth" are the reserved height and width
%		of the figure, i.e., how big TeX actually thinks
%		the figure is.  They default to "width" and "height".
%	The "clip" option ensures that no portion of the figure will
%		appear outside its BoundingBox.  "clip=" is a switch and
%		takes no value, but the `=' must be present.
%	The "angle" option specifies the angle of rotation (degrees, ccw).
%	The "silent" option makes \psfig work silently.
%
% ======================================================================
% check to see if macros already loaded in (maybe some other file says
% "\input psfig") ...
\ifx\undefined\psfig\else\endinput\fi
%
% from a suggestion by eijkhout@csrd.uiuc.edu to allow
% loading as a style file. Changed to avoid problems
% with amstex per suggestion by jbence@math.ucla.edu

\let\LaTeXAtSign=\@
\let\@=\relax
\edef\psfigRestoreAt{\catcode`\@=\number\catcode`@\relax}
%\edef\psfigRestoreAt{\catcode`@=\number\catcode`@\relax}
\catcode`\@=11\relax
\newwrite\@unused
\def\ps@typeout#1{{\let\protect\string\immediate\write\@unused{#1}}}

\def\DvipsDriver{
	\ps@typeout{psfig/tex \PsfigVersion -dvips}
\def\PsfigSpecials{\DvipsSpecials} 	\def\ps@dir{/}
\def\ps@predir{} }
\def\OzTeXDriver{
	\ps@typeout{psfig/tex \PsfigVersion -oztex}
	\def\PsfigSpecials{\OzTeXSpecials}
	\def\ps@dir{:}
	\def\ps@predir{:}
	\catcode`\^^J=5
}

%% Here's how you define your figure path.  Should be set up with null
%% default and a user useable definition.

\def\figurepath{./:}
\def\psfigurepath#1{\edef\figurepath{#1:}}

%%% inserted for Searching Unixpaths
%%% (the path must end with :)
%%% (call: \DoPaths\figurepath )
%%%------------------------------------------------------
\def\DoPaths#1{\expandafter\EachPath#1\stoplist}
%
\def\leer{}
\def\EachPath#1:#2\stoplist{% #1 part of the list (delimiter :)
  \ExistsFile{#1}{\SearchedFile}
  \ifx#2\leer
  \else
    \expandafter\EachPath#2\stoplist
  \fi}
%
% exists the file (does not work for directories!)
%
\def\ps@dir{/}
\def\ExistsFile#1#2{%
   \openin1=\ps@predir#1\ps@dir#2
   \ifeof1
       \closein1
       %\ps@typeout{...not: \ps@predir#1\ps@dir#2}
   \else
       \closein1
       %\ps@typeout{...in:  \ps@predir#1\ps@dir#2}
        \ifx\ps@founddir\leer
          %\ps@typeout{set founddir #1}
           \edef\ps@founddir{#1}
        \fi
   \fi}
%------------------------------------------------------
%
% Get dir in path or error
%
\def\get@dir#1{%
  \def\ps@founddir{}
  \def\SearchedFile{#1}
  \DoPaths\figurepath
%  \fi
}
%------------------------------------------------------
%%% END of Searching Unixpaths


%
% @psdo control structure -- similar to Latex @for.
% I redefined these with different names so that psfig can
% be used with TeX as well as LaTeX, and so that it will not 
% be vunerable to future changes in LaTeX's internal
% control structure,
%
\def\@nnil{\@nil}
\def\@empty{}
\def\@psdonoop#1\@@#2#3{}
\def\@psdo#1:=#2\do#3{\edef\@psdotmp{#2}\ifx\@psdotmp\@empty \else
    \expandafter\@psdoloop#2,\@nil,\@nil\@@#1{#3}\fi}
\def\@psdoloop#1,#2,#3\@@#4#5{\def#4{#1}\ifx #4\@nnil \else
       #5\def#4{#2}\ifx #4\@nnil \else#5\@ipsdoloop #3\@@#4{#5}\fi\fi}
\def\@ipsdoloop#1,#2\@@#3#4{\def#3{#1}\ifx #3\@nnil 
       \let\@nextwhile=\@psdonoop \else
      #4\relax\let\@nextwhile=\@ipsdoloop\fi\@nextwhile#2\@@#3{#4}}
\def\@tpsdo#1:=#2\do#3{\xdef\@psdotmp{#2}\ifx\@psdotmp\@empty \else
    \@tpsdoloop#2\@nil\@nil\@@#1{#3}\fi}
\def\@tpsdoloop#1#2\@@#3#4{\def#3{#1}\ifx #3\@nnil 
       \let\@nextwhile=\@psdonoop \else
      #4\relax\let\@nextwhile=\@tpsdoloop\fi\@nextwhile#2\@@#3{#4}}
% 
% \fbox is defined in latex.tex; so if \fbox is undefined, assume that
% we are not in LaTeX.
% Perhaps this could be done better???
\ifx\undefined\fbox
% \fbox code from modified slightly from LaTeX
\newdimen\fboxrule
\newdimen\fboxsep
\newdimen\ps@tempdima
\newbox\ps@tempboxa
\fboxsep = 3pt
\fboxrule = .4pt
\long\def\fbox#1{\leavevmode\setbox\ps@tempboxa\hbox{#1}\ps@tempdima\fboxrule
    \advance\ps@tempdima \fboxsep \advance\ps@tempdima \dp\ps@tempboxa
   \hbox{\lower \ps@tempdima\hbox
  {\vbox{\hrule height \fboxrule
          \hbox{\vrule width \fboxrule \hskip\fboxsep
          \vbox{\vskip\fboxsep \box\ps@tempboxa\vskip\fboxsep}\hskip 
                 \fboxsep\vrule width \fboxrule}
                 \hrule height \fboxrule}}}}
\fi
%
%%%%%%%%%%%%%%%%%%%%%%%%%%%%%%%%%%%%%%%%%%%%%%%%%%%%%%%%%%%%%%%%%%%
% file reading stuff from epsf.tex
%   EPSF.TEX macro file:
%   Written by Tomas Rokicki of Radical Eye Software, 29 Mar 1989.
%   Revised by Don Knuth, 3 Jan 1990.
%   Revised by Tomas Rokicki to accept bounding boxes with no
%      space after the colon, 18 Jul 1990.
%   Portions modified/removed for use in PSFIG package by
%      J. Daniel Smith, 9 October 1990.
%
\newread\ps@stream
\newif\ifnot@eof       % continue looking for the bounding box?
\newif\if@noisy        % report what you're making?
\newif\if@atend        % %%BoundingBox: has (at end) specification
\newif\if@psfile       % does this look like a PostScript file?
%
% PostScript files should start with `%!'
%
{\catcode`\%=12\global\gdef\epsf@start{%!}}
\def\epsf@PS{PS}
%
\def\epsf@getbb#1{%
%
%   The first thing we need to do is to open the
%   PostScript file, if possible.
%
\openin\ps@stream=\ps@predir#1
\ifeof\ps@stream\ps@typeout{Error, File #1 not found}\else
%
%   Okay, we got it. Now we'll scan lines until we find one that doesn't
%   start with %. We're looking for the bounding box comment.
%
   {\not@eoftrue \chardef\other=12
    \def\do##1{\catcode`##1=\other}\dospecials \catcode`\ =10
    \loop
       \if@psfile
	  \read\ps@stream to \epsf@fileline
       \else{
	  \obeyspaces
          \read\ps@stream to \epsf@tmp\global\let\epsf@fileline\epsf@tmp}
       \fi
       \ifeof\ps@stream\not@eoffalse\else
%
%   Check the first line for `%!'.  Issue a warning message if its not
%   there, since the file might not be a PostScript file.
%
       \if@psfile\else
       \expandafter\epsf@test\epsf@fileline:. \\%
       \fi
%
%   We check to see if the first character is a % sign;
%   if so, we look further and stop only if the line begins with
%   `%%BoundingBox:' and the `(atend)' specification was not found.
%   That is, the only way to stop is when the end of file is reached,
%   or a `%%BoundingBox: llx lly urx ury' line is found.
%
          \expandafter\epsf@aux\epsf@fileline:. \\%
       \fi
   \ifnot@eof\repeat
   }\closein\ps@stream\fi}%
%
% This tests if the file we are reading looks like a PostScript file.
%
\long\def\epsf@test#1#2#3:#4\\{\def\epsf@testit{#1#2}
			\ifx\epsf@testit\epsf@start\else
\ps@typeout{Warning! File does not start with `\epsf@start'.  It may not be a PostScript file.}
			\fi
			\@psfiletrue} % don't test after 1st line
%
%   We still need to define the tricky \epsf@aux macro. This requires
%   a couple of magic constants for comparison purposes.
%
{\catcode`\%=12\global\let\epsf@percent=%\global\def\epsf@bblit{%BoundingBox}}
%
%
%   So we're ready to check for `%BoundingBox:' and to grab the
%   values if they are found.  We continue searching if `(at end)'
%   was found after the `%BoundingBox:'.
%
\long\def\epsf@aux#1#2:#3\\{\ifx#1\epsf@percent
   \def\epsf@testit{#2}\ifx\epsf@testit\epsf@bblit
	\@atendfalse
        \epsf@atend #3 . \\%
	\if@atend	
	   \if@verbose{
		\ps@typeout{psfig: found `(atend)'; continuing search}
	   }\fi
        \else
        \epsf@grab #3 . . . \\%
        \not@eoffalse
        \global\no@bbfalse
        \fi
   \fi\fi}%
%
%   Here we grab the values and stuff them in the appropriate definitions.
%
\def\epsf@grab #1 #2 #3 #4 #5\\{%
   \global\def\epsf@llx{#1}\ifx\epsf@llx\empty
      \epsf@grab #2 #3 #4 #5 .\\\else
   \global\def\epsf@lly{#2}%
   \global\def\epsf@urx{#3}\global\def\epsf@ury{#4}\fi}%
%
% Determine if the stuff following the %%BoundingBox is `(atend)'
% J. Daniel Smith.  Copied from \epsf@grab above.
%
\def\epsf@atendlit{(atend)} 
\def\epsf@atend #1 #2 #3\\{%
   \def\epsf@tmp{#1}\ifx\epsf@tmp\empty
      \epsf@atend #2 #3 .\\\else
   \ifx\epsf@tmp\epsf@atendlit\@atendtrue\fi\fi}


% End of file reading stuff from epsf.tex
%%%%%%%%%%%%%%%%%%%%%%%%%%%%%%%%%%%%%%%%%%%%%%%%%%%%%%%%%%%%%%%%%%%

%%%%%%%%%%%%%%%%%%%%%%%%%%%%%%%%%%%%%%%%%%%%%%%%%%%%%%%%%%%%%%%%%%%
% trigonometry stuff from "trig.tex"
\chardef\psletter = 11 % won't conflict with \begin{letter} now...
\chardef\other = 12

\newif \ifdebug %%% turn me on to see TeX hard at work ...
\newif\ifc@mpute %%% don't need to compute some values
\c@mputetrue % but assume that we do

\let\then = \relax
\def\r@dian{pt }
\let\r@dians = \r@dian
\let\dimensionless@nit = \r@dian
\let\dimensionless@nits = \dimensionless@nit
\def\internal@nit{sp }
\let\internal@nits = \internal@nit
\newif\ifstillc@nverging
\def \Mess@ge #1{\ifdebug \then \message {#1} \fi}

{ %%% Things that need abnormal catcodes %%%
	\catcode `\@ = \psletter
	\gdef \nodimen {\expandafter \n@dimen \the \dimen}
	\gdef \term #1 #2 #3%
	       {\edef \t@ {\the #1}%%% freeze parameter 1 (count, by value)
		\edef \t@@ {\expandafter \n@dimen \the #2\r@dian}%
				   %%% freeze parameter 2 (dimen, by value)
		\t@rm {\t@} {\t@@} {#3}%
	       }
	\gdef \t@rm #1 #2 #3%
	       {{%
		\count 0 = 0
		\dimen 0 = 1 \dimensionless@nit
		\dimen 2 = #2\relax
		\Mess@ge {Calculating term #1 of \nodimen 2}%
		\loop
		\ifnum	\count 0 < #1
		\then	\advance \count 0 by 1
			\Mess@ge {Iteration \the \count 0 \space}%
			\Multiply \dimen 0 by {\dimen 2}%
			\Mess@ge {After multiplication, term = \nodimen 0}%
			\Divide \dimen 0 by {\count 0}%
			\Mess@ge {After division, term = \nodimen 0}%
		\repeat
		\Mess@ge {Final value for term #1 of 
				\nodimen 2 \space is \nodimen 0}%
		\xdef \Term {#3 = \nodimen 0 \r@dians}%
		\aftergroup \Term
	       }}
	\catcode `\p = \other
	\catcode `\t = \other
	\gdef \n@dimen #1pt{#1} %%% throw away the ``pt''
}

\def \Divide #1by #2{\divide #1 by #2} %%% just a synonym

\def \Multiply #1by #2%%% allows division of a dimen by a dimen
       {{%%% should really freeze parameter 2 (dimen, passed by value)
	\count 0 = #1\relax
	\count 2 = #2\relax
	\count 4 = 65536
	\Mess@ge {Before scaling, count 0 = \the \count 0 \space and
			count 2 = \the \count 2}%
	\ifnum	\count 0 > 32767 %%% do our best to avoid overflow
	\then	\divide \count 0 by 4
		\divide \count 4 by 4
	\else	\ifnum	\count 0 < -32767
		\then	\divide \count 0 by 4
			\divide \count 4 by 4
		\else
		\fi
	\fi
	\ifnum	\count 2 > 32767 %%% while retaining reasonable accuracy
	\then	\divide \count 2 by 4
		\divide \count 4 by 4
	\else	\ifnum	\count 2 < -32767
		\then	\divide \count 2 by 4
			\divide \count 4 by 4
		\else
		\fi
	\fi
	\multiply \count 0 by \count 2
	\divide \count 0 by \count 4
	\xdef \product {#1 = \the \count 0 \internal@nits}%
	\aftergroup \product
       }}

\def\r@duce{\ifdim\dimen0 > 90\r@dian \then   % sin(x+90) = sin(180-x)
		\multiply\dimen0 by -1
		\advance\dimen0 by 180\r@dian
		\r@duce
	    \else \ifdim\dimen0 < -90\r@dian \then  % sin(-x) = sin(360+x)
		\advance\dimen0 by 360\r@dian
		\r@duce
		\fi
	    \fi}

\def\Sine#1%
       {{%
	\dimen 0 = #1 \r@dian
	\r@duce
	\ifdim\dimen0 = -90\r@dian \then
	   \dimen4 = -1\r@dian
	   \c@mputefalse
	\fi
	\ifdim\dimen0 = 90\r@dian \then
	   \dimen4 = 1\r@dian
	   \c@mputefalse
	\fi
	\ifdim\dimen0 = 0\r@dian \then
	   \dimen4 = 0\r@dian
	   \c@mputefalse
	\fi
%
	\ifc@mpute \then
        	% convert degrees to radians
		\divide\dimen0 by 180
		\dimen0=3.141592654\dimen0
%
		\dimen 2 = 3.1415926535897963\r@dian %%% a well-known constant
		\divide\dimen 2 by 2 %%% we only deal with -pi/2 : pi/2
		\Mess@ge {Sin: calculating Sin of \nodimen 0}%
		\count 0 = 1 %%% see power-series expansion for sine
		\dimen 2 = 1 \r@dian %%% ditto
		\dimen 4 = 0 \r@dian %%% ditto
		\loop
			\ifnum	\dimen 2 = 0 %%% then we've done
			\then	\stillc@nvergingfalse 
			\else	\stillc@nvergingtrue
			\fi
			\ifstillc@nverging %%% then calculate next term
			\then	\term {\count 0} {\dimen 0} {\dimen 2}%
				\advance \count 0 by 2
				\count 2 = \count 0
				\divide \count 2 by 2
				\ifodd	\count 2 %%% signs alternate
				\then	\advance \dimen 4 by \dimen 2
				\else	\advance \dimen 4 by -\dimen 2
				\fi
		\repeat
	\fi		
			\xdef \sine {\nodimen 4}%
       }}

% Now the Cosine can be calculated easily by calling \Sine
\def\Cosine#1{\ifx\sine\UnDefined\edef\Savesine{\relax}\else
		             \edef\Savesine{\sine}\fi
	{\dimen0=#1\r@dian\advance\dimen0 by 90\r@dian
	 \Sine{\nodimen 0}
	 \xdef\cosine{\sine}
	 \xdef\sine{\Savesine}}}	      
% end of trig stuff
%%%%%%%%%%%%%%%%%%%%%%%%%%%%%%%%%%%%%%%%%%%%%%%%%%%%%%%%%%%%%%%%%%%%

\def\psdraft{
	\def\@psdraft{0}
	%\ps@typeout{draft level now is \@psdraft \space . }
}
\def\psfull{
	\def\@psdraft{100}
	%\ps@typeout{draft level now is \@psdraft \space . }
}

\psfull

\newif\if@scalefirst
\def\psscalefirst{\@scalefirsttrue}
\def\psrotatefirst{\@scalefirstfalse}
\psrotatefirst

\newif\if@draftbox
\def\psnodraftbox{
	\@draftboxfalse
}
\def\psdraftbox{
	\@draftboxtrue
}
\@draftboxtrue

\newif\if@prologfile
\newif\if@postlogfile
\def\pssilent{
	\@noisyfalse
}
\def\psnoisy{
	\@noisytrue
}
\psnoisy
%%% These are for the option list.
%%% A specification of the form a = b maps to calling \@p@@sa{b}
\newif\if@bbllx
\newif\if@bblly
\newif\if@bburx
\newif\if@bbury
\newif\if@height
\newif\if@width
\newif\if@rheight
\newif\if@rwidth
\newif\if@angle
\newif\if@clip
\newif\if@verbose
\def\@p@@sclip#1{\@cliptrue}
%
%
\newif\if@decmpr
%
\def\@p@@sfigure#1{\def\@p@sfile{null}\def\@p@sbbfile{null}\@decmprfalse
   % look directly for file (e.g. absolute path)
   \openin1=\ps@predir#1
   \ifeof1
	\closein1
	% failed, search directories for file
	\get@dir{#1}
	\ifx\ps@founddir\leer
		% failed, search directly for file.bb
		\openin1=\ps@predir#1.bb
		\ifeof1
			\closein1
			% failed, search directories for file.bb
			\get@dir{#1.bb}
			\ifx\ps@founddir\leer
				% failed, lose.
				\ps@typeout{Can't find #1 in \figurepath}
			\else
				% found file.bb in search dir
				\@decmprtrue
				\def\@p@sfile{\ps@founddir\ps@dir#1}
				\def\@p@sbbfile{\ps@founddir\ps@dir#1.bb}
			\fi
		\else
			\closein1
			%found file.bb directly
			\@decmprtrue
			\def\@p@sfile{#1}
			\def\@p@sbbfile{#1.bb}
		\fi
	\else
		% found file in search dir
		\def\@p@sfile{\ps@founddir\ps@dir#1}
		\def\@p@sbbfile{\ps@founddir\ps@dir#1}
	\fi
   \else
	% found file directly
	\closein1
	\def\@p@sfile{#1}
	\def\@p@sbbfile{#1}
   \fi
}
%
%
%
\def\@p@@sfile#1{\@p@@sfigure{#1}}
%
\def\@p@@sbbllx#1{
		%\ps@typeout{bbllx is #1}
		\@bbllxtrue
		\dimen100=#1
		\edef\@p@sbbllx{\number\dimen100}
}
\def\@p@@sbblly#1{
		%\ps@typeout{bblly is #1}
		\@bbllytrue
		\dimen100=#1
		\edef\@p@sbblly{\number\dimen100}
}
\def\@p@@sbburx#1{
		%\ps@typeout{bburx is #1}
		\@bburxtrue
		\dimen100=#1
		\edef\@p@sbburx{\number\dimen100}
}
\def\@p@@sbbury#1{
		%\ps@typeout{bbury is #1}
		\@bburytrue
		\dimen100=#1
		\edef\@p@sbbury{\number\dimen100}
}
\def\@p@@sheight#1{
		\@heighttrue
		\dimen100=#1
   		\edef\@p@sheight{\number\dimen100}
		%\ps@typeout{Height is \@p@sheight}
}
\def\@p@@swidth#1{
		%\ps@typeout{Width is #1}
		\@widthtrue
		\dimen100=#1
		\edef\@p@swidth{\number\dimen100}
}
\def\@p@@srheight#1{
		%\ps@typeout{Reserved height is #1}
		\@rheighttrue
		\dimen100=#1
		\edef\@p@srheight{\number\dimen100}
}
\def\@p@@srwidth#1{
		%\ps@typeout{Reserved width is #1}
		\@rwidthtrue
		\dimen100=#1
		\edef\@p@srwidth{\number\dimen100}
}
\def\@p@@sangle#1{
		%\ps@typeout{Rotation is #1}
		\@angletrue
%		\dimen100=#1
		\edef\@p@sangle{#1} %\number\dimen100}
}
\def\@p@@ssilent#1{ 
		\@verbosefalse
}
\def\@p@@sprolog#1{\@prologfiletrue\def\@prologfileval{#1}}
\def\@p@@spostlog#1{\@postlogfiletrue\def\@postlogfileval{#1}}
\def\@cs@name#1{\csname #1\endcsname}
\def\@setparms#1=#2,{\@cs@name{@p@@s#1}{#2}}
%
% initialize the defaults (size the size of the figure)
%
\def\ps@init@parms{
		\@bbllxfalse \@bbllyfalse
		\@bburxfalse \@bburyfalse
		\@heightfalse \@widthfalse
		\@rheightfalse \@rwidthfalse
		\def\@p@sbbllx{}\def\@p@sbblly{}
		\def\@p@sbburx{}\def\@p@sbbury{}
		\def\@p@sheight{}\def\@p@swidth{}
		\def\@p@srheight{}\def\@p@srwidth{}
		\def\@p@sangle{0}
		\def\@p@sfile{} \def\@p@sbbfile{}
		\def\@p@scost{10}
		\def\@sc{}
		\@prologfilefalse
		\@postlogfilefalse
		\@clipfalse
		\if@noisy
			\@verbosetrue
		\else
			\@verbosefalse
		\fi
}
%
% Go through the options setting things up.
%
\def\parse@ps@parms#1{
	 	\@psdo\@psfiga:=#1\do
		   {\expandafter\@setparms\@psfiga,}}
%
% Compute bb height and width
%
\newif\ifno@bb
\def\bb@missing{
	\if@verbose{
		\ps@typeout{psfig: searching \@p@sbbfile \space  for bounding box}
	}\fi
	\no@bbtrue
	\epsf@getbb{\@p@sbbfile}
        \ifno@bb \else \bb@cull\epsf@llx\epsf@lly\epsf@urx\epsf@ury\fi
}	
\def\bb@cull#1#2#3#4{
	\dimen100=#1 bp\edef\@p@sbbllx{\number\dimen100}
	\dimen100=#2 bp\edef\@p@sbblly{\number\dimen100}
	\dimen100=#3 bp\edef\@p@sbburx{\number\dimen100}
	\dimen100=#4 bp\edef\@p@sbbury{\number\dimen100}
	\no@bbfalse
}
% rotate point (#1,#2) about (0,0).
% The sine and cosine of the angle are already stored in \sine and
% \cosine.  The result is placed in (\p@intvaluex, \p@intvaluey).
\newdimen\p@intvaluex
\newdimen\p@intvaluey
\def\rotate@#1#2{{\dimen0=#1 sp\dimen1=#2 sp
%            	calculate x' = x \cos\theta - y \sin\theta
		  \global\p@intvaluex=\cosine\dimen0
		  \dimen3=\sine\dimen1
		  \global\advance\p@intvaluex by -\dimen3
% 		calculate y' = x \sin\theta + y \cos\theta
		  \global\p@intvaluey=\sine\dimen0
		  \dimen3=\cosine\dimen1
		  \global\advance\p@intvaluey by \dimen3
		  }}
\def\compute@bb{
		\no@bbfalse
		\if@bbllx \else \no@bbtrue \fi
		\if@bblly \else \no@bbtrue \fi
		\if@bburx \else \no@bbtrue \fi
		\if@bbury \else \no@bbtrue \fi
		\ifno@bb \bb@missing \fi
		\ifno@bb \ps@typeout{FATAL ERROR: no bb supplied or found}
			\no-bb-error
		\fi
		%
%\ps@typeout{BB: \@p@sbbllx, \@p@sbblly, \@p@sbburx, \@p@sbbury} 
%
% store height/width of original (unrotated) bounding box
		\count203=\@p@sbburx
		\count204=\@p@sbbury
		\advance\count203 by -\@p@sbbllx
		\advance\count204 by -\@p@sbblly
		\edef\ps@bbw{\number\count203}
		\edef\ps@bbh{\number\count204}
		%\ps@typeout{ psbbh = \ps@bbh, psbbw = \ps@bbw }
		\if@angle 
			\Sine{\@p@sangle}\Cosine{\@p@sangle}
	        	{\dimen100=\maxdimen\xdef\r@p@sbbllx{\number\dimen100}
					    \xdef\r@p@sbblly{\number\dimen100}
			                    \xdef\r@p@sbburx{-\number\dimen100}
					    \xdef\r@p@sbbury{-\number\dimen100}}
%
% Need to rotate all four points and take the X-Y extremes of the new
% points as the new bounding box.
                        \def\minmaxtest{
			   \ifnum\number\p@intvaluex<\r@p@sbbllx
			      \xdef\r@p@sbbllx{\number\p@intvaluex}\fi
			   \ifnum\number\p@intvaluex>\r@p@sbburx
			      \xdef\r@p@sbburx{\number\p@intvaluex}\fi
			   \ifnum\number\p@intvaluey<\r@p@sbblly
			      \xdef\r@p@sbblly{\number\p@intvaluey}\fi
			   \ifnum\number\p@intvaluey>\r@p@sbbury
			      \xdef\r@p@sbbury{\number\p@intvaluey}\fi
			   }
%			lower left
			\rotate@{\@p@sbbllx}{\@p@sbblly}
			\minmaxtest
%			upper left
			\rotate@{\@p@sbbllx}{\@p@sbbury}
			\minmaxtest
%			lower right
			\rotate@{\@p@sbburx}{\@p@sbblly}
			\minmaxtest
%			upper right
			\rotate@{\@p@sbburx}{\@p@sbbury}
			\minmaxtest
			\edef\@p@sbbllx{\r@p@sbbllx}\edef\@p@sbblly{\r@p@sbblly}
			\edef\@p@sbburx{\r@p@sbburx}\edef\@p@sbbury{\r@p@sbbury}
%\ps@typeout{rotated BB: \r@p@sbbllx, \r@p@sbblly, \r@p@sbburx, \r@p@sbbury}
		\fi
		\count203=\@p@sbburx
		\count204=\@p@sbbury
		\advance\count203 by -\@p@sbbllx
		\advance\count204 by -\@p@sbblly
		\edef\@bbw{\number\count203}
		\edef\@bbh{\number\count204}
		%\ps@typeout{ bbh = \@bbh, bbw = \@bbw }
}
%
% \in@hundreds performs #1 * (#2 / #3) correct to the hundreds,
%	then leaves the result in @result
%
\def\in@hundreds#1#2#3{\count240=#2 \count241=#3
		     \count100=\count240	% 100 is first digit #2/#3
		     \divide\count100 by \count241
		     \count101=\count100
		     \multiply\count101 by \count241
		     \advance\count240 by -\count101
		     \multiply\count240 by 10
		     \count101=\count240	%101 is second digit of #2/#3
		     \divide\count101 by \count241
		     \count102=\count101
		     \multiply\count102 by \count241
		     \advance\count240 by -\count102
		     \multiply\count240 by 10
		     \count102=\count240	% 102 is the third digit
		     \divide\count102 by \count241
		     \count200=#1\count205=0
		     \count201=\count200
			\multiply\count201 by \count100
		 	\advance\count205 by \count201
		     \count201=\count200
			\divide\count201 by 10
			\multiply\count201 by \count101
			\advance\count205 by \count201
			%
		     \count201=\count200
			\divide\count201 by 100
			\multiply\count201 by \count102
			\advance\count205 by \count201
			%
		     \edef\@result{\number\count205}
}
\def\compute@wfromh{
		% computing : width = height * (bbw / bbh)
		\in@hundreds{\@p@sheight}{\@bbw}{\@bbh}
		%\ps@typeout{ \@p@sheight * \@bbw / \@bbh, = \@result }
		\edef\@p@swidth{\@result}
		%\ps@typeout{w from h: width is \@p@swidth}
}
\def\compute@hfromw{
		% computing : height = width * (bbh / bbw)
	        \in@hundreds{\@p@swidth}{\@bbh}{\@bbw}
		%\ps@typeout{ \@p@swidth * \@bbh / \@bbw = \@result }
		\edef\@p@sheight{\@result}
		%\ps@typeout{h from w : height is \@p@sheight}
}
\def\compute@handw{
		\if@height 
			\if@width
			\else
				\compute@wfromh
			\fi
		\else 
			\if@width
				\compute@hfromw
			\else
				\edef\@p@sheight{\@bbh}
				\edef\@p@swidth{\@bbw}
			\fi
		\fi
}
\def\compute@resv{
		\if@rheight \else \edef\@p@srheight{\@p@sheight} \fi
		\if@rwidth \else \edef\@p@srwidth{\@p@swidth} \fi
		%\ps@typeout{rheight = \@p@srheight, rwidth = \@p@srwidth}
}
%		
% Compute any missing values
\def\compute@sizes{
	\compute@bb
	\if@scalefirst\if@angle
% at this point the bounding box has been adjsuted correctly for
% rotation.  PSFIG does all of its scaling using \@bbh and \@bbw.  If
% a width= or height= was specified along with \psscalefirst, then the
% width=/height= value needs to be adjusted to match the new (rotated)
% bounding box size (specifed in \@bbw and \@bbh).
%    \ps@bbw       width=
%    -------  =  ---------- 
%    \@bbw       new width=
% so `new width=' = (width= * \@bbw) / \ps@bbw; where \ps@bbw is the
% width of the original (unrotated) bounding box.
	\if@width
	   \in@hundreds{\@p@swidth}{\@bbw}{\ps@bbw}
	   \edef\@p@swidth{\@result}
	\fi
	\if@height
	   \in@hundreds{\@p@sheight}{\@bbh}{\ps@bbh}
	   \edef\@p@sheight{\@result}
	\fi
	\fi\fi
	\compute@handw
	\compute@resv}
%
%
%
\def\OzTeXSpecials{
	\special{empty.ps /@isp {true} def}
	\special{empty.ps \@p@swidth \space \@p@sheight \space
			\@p@sbbllx \space \@p@sbblly \space
			\@p@sbburx \space \@p@sbbury \space
			startTexFig \space }
	\if@clip{
		\if@verbose{
			\ps@typeout{(clip)}
		}\fi
		\special{empty.ps doclip \space }
	}\fi
	\if@angle{
		\if@verbose{
			\ps@typeout{(rotate)}
		}\fi
		\special {empty.ps \@p@sangle \space rotate \space} 
	}\fi
	\if@prologfile
	    \special{\@prologfileval \space } \fi
	\if@decmpr{
		\if@verbose{
			\ps@typeout{psfig: Compression not available
			in OzTeX version \space }
		}\fi
	}\else{
		\if@verbose{
			\ps@typeout{psfig: including \@p@sfile \space }
		}\fi
		\special{epsf=\@p@sfile \space }
	}\fi
	\if@postlogfile
	    \special{\@postlogfileval \space } \fi
	\special{empty.ps /@isp {false} def}
}
\def\DvipsSpecials{
	%
	\special{ps::[begin] 	\@p@swidth \space \@p@sheight \space
			\@p@sbbllx \space \@p@sbblly \space
			\@p@sbburx \space \@p@sbbury \space
			startTexFig \space }
	\if@clip{
		\if@verbose{
			\ps@typeout{(clip)}
		}\fi
		\special{ps:: doclip \space }
	}\fi
	\if@angle
		\if@verbose{
			\ps@typeout{(clip)}
		}\fi
		\special {ps:: \@p@sangle \space rotate \space} 
	\fi
	\if@prologfile
	    \special{ps: plotfile \@prologfileval \space } \fi
	\if@decmpr{
		\if@verbose{
			\ps@typeout{psfig: including \@p@sfile.Z \space }
		}\fi
		\special{ps: plotfile "`zcat \@p@sfile.Z" \space }
	}\else{
		\if@verbose{
			\ps@typeout{psfig: including \@p@sfile \space }
		}\fi
		\special{ps: plotfile \@p@sfile \space }
	}\fi
	\if@postlogfile
	    \special{ps: plotfile \@postlogfileval \space } \fi
	\special{ps::[end] endTexFig \space }
}
%
% \psfig
% usage : \psfig{file=, height=, width=, bbllx=, bblly=, bburx=, bbury=,
%			rheight=, rwidth=, clip=}
%
% "clip=" is a switch and takes no value, but the `=' must be present.
\def\psfig#1{\vbox {
	% do a zero width hard space so that a single
	% \psfig in a centering enviornment will behave nicely
	%{\setbox0=\hbox{\ }\ \hskip-\wd0}
	%
	\ps@init@parms
	\parse@ps@parms{#1}
	\compute@sizes
	%
	\ifnum\@p@scost<\@psdraft{
		\PsfigSpecials 
		% Create the vbox to reserve the space for the figure.
		\vbox to \@p@srheight sp{
		% 1/92 TJD Changed from "true sp" to "sp" for magnification.
			\hbox to \@p@srwidth sp{
				\hss
			}
		\vss
		}
	}\else{
		% draft figure, just reserve the space and print the
		% path name.
		\if@draftbox{		
			% Verbose draft: print file name in box
			% 10/93 TJD changed to fbox from frame
			\hbox{\fbox{\vbox to \@p@srheight sp{
			\vss
			\hbox to \@p@srwidth sp{ \hss 
			        % 10/93 TJD deleted to avoid ``_'' problems
				% \@p@sfile
			 \hss }
			\vss
			}}}
		}\else{
			% Non-verbose draft
			\vbox to \@p@srheight sp{
			\vss
			\hbox to \@p@srwidth sp{\hss}
			\vss
			}
		}\fi	



	}\fi
}}
\psfigRestoreAt
\setDriver
\let\@=\LaTeXAtSign




% \input epsf %% at top of file
\jairheading{2}{1994}{1-32}{4/94}{8/94}
\ShortHeadings{Induction of Oblique Decision Trees}%
{Murthy, Kasif \& Salzberg}
\firstpageno{1}

\title {A System for Induction of Oblique Decision Trees}
\author {\name Sreerama K. Murthy \email murthy@cs.jhu.edu \\
         \name Simon Kasif \email kasif@cs.jhu.edu \\
         \name Steven Salzberg \email salzberg@cs.jhu.edu \\
         \addr Department of Computer Science \\
         Johns Hopkins University,
         Baltimore, MD 21218 USA}
\begin{document}
\maketitle

\begin{abstract}
This article describes a new system for induction of oblique decision
trees.  This system, OC1, combines deterministic hill-climbing with
two forms of randomization to find a good oblique split (in the form
of a hyperplane) at each node of a decision tree.  Oblique decision
tree methods are tuned especially for domains in which the attributes
are numeric, although they can be adapted to symbolic or mixed
symbolic/numeric attributes.  We present extensive empirical studies,
using both real and artificial data, that analyze OC1's ability to
construct oblique trees that are smaller and more accurate than their
axis-parallel counterparts.  We also examine the benefits of
randomization for the construction of oblique decision trees.
\end{abstract}

\section{Introduction}
\label{section:intro}

Current data collection technology provides a unique challenge and
opportunity for automated machine learning techniques.  The advent of
major scientific projects such as the Human Genome Project, the Hubble
Space Telescope, and the human brain mapping initiative are generating
enormous amounts of data on a daily basis.  These streams of data
require automated methods to analyze, filter, and classify them before
presenting them in digested form to a domain scientist.  Decision
trees are a particularly useful tool in this context because they
perform classification by a sequence of simple, easy-to-understand
tests whose semantics is intuitively clear to domain experts.
Decision trees have been used for classification and other tasks
since the 1960s \cite{moret/82,safavin/landgrebe/91}.  In the 1980's,
Breiman et al.'s book on classification and regression trees (CART)
and Quinlan's work on ID3 \cite{quinlan/83,quinlan/86} provided the
foundations for what has become a large body of research on one of the
central techniques of experimental machine learning.

Many variants of decision tree (DT) algorithms have been introduced in
the last decade.  Much of this work has concentrated on decision trees
in which each node checks the value of a single attribute
\cite{breiman/etal/84,quinlan/86,quinlan/93}.  Quinlan initially
proposed decision trees for classification in domains with
symbolic-valued attributes
\citeyear{quinlan/86}, and later extended them to numeric domains
\citeyear{quinlan/87b}.  When the attributes are numeric, the tests 
have the form $x_i > k$, where $x_i$ is one of the attributes of an
example and $k$ is a constant.  This class of decision trees may be
called {\it axis-parallel}, because the tests at each node are
equivalent to axis-parallel hyperplanes in the attribute space.  An
example of such a decision tree is given in
Figure~\ref{figure:aptree}, which shows both a tree and the
partitioning it creates in a 2-D attribute space.

\begin{figure}
\vspace{2.0in}
\special{psfile="aptree.ps" hoffset=20 voffset=-330}
\caption{The left side of the figure shows a simple axis-parallel tree 
that uses two attributes.  The right side shows the partitioning
that this tree creates in the attribute space.}
\label{figure:aptree}
\vspace*{-.2in} 
\end{figure}

Researchers have also studied decision trees in which the test at a
node uses boolean combinations of attributes
\cite{pagallo/90,pagallo/haussler/90,sahami/93} and linear combinations
of attributes (see Section \ref{section:oblique}).  Different methods
for measuring the goodness of decision tree nodes, as well as
techniques for pruning a tree to reduce {\it overfitting} and increase
accuracy have also been explored, and will be discussed in later
sections.

In this paper, we examine decision trees that test a linear
combination of the attributes at each internal node.  More precisely,
let an example take the form $X = x_1,x_2,\ldots,x_d,C_j$ where $C_j$
is a class label and the $x_i$'s are real-valued
attributes.\footnote{The constraint that $x_1,\ldots,x_d$ are
real-valued does not necessarily restrict oblique decision trees to
numeric domains.  Several researchers have studied the problem of
converting symbolic (unordered) domains to numeric (ordered) domains
and vice versa; e.g., \cite{breiman/etal/84,hampson/volper/86,%
utgoff/brodley/90,deMerckt/92,deMerckt/93}.  To keep the discussion
simple, however, we will assume that all attributes have numeric
values.} The test at each node will then have the form:
\begin{equation}
\label{equation:hyperplane}
 \sum_{i=1}^{d}{a_ix_i} + a_{d+1} > 0  
\end{equation}
where $a_1,\ldots,a_{d+1}$ are real-valued coefficients.  Because
these tests are equivalent to hyperplanes at an oblique orientation to
the axes, we call this class of decision trees {\em oblique} decision
trees.  (Trees of this form have also been called ``multivariate''
\cite{brodley/utgoff/94}.  We prefer the term ``oblique'' because
``multivariate'' includes non-linear combinations of the variables,
i.e., curved surfaces.  Our trees contain only linear tests.)  It is
clear that these are simply a more general form of axis-parallel
trees, since by setting $a_i = 0$ for all coefficients but one, the
test in Eq.~\ref{equation:hyperplane} becomes the familiar univariate
test.  Note that oblique decision trees produce polygonal (polyhedral)
partitionings of the attribute space, while axis-parallel trees
produce partitionings in the form of hyper-rectangles that are
parallel to the feature axes.

It should be intuitively clear that when the underlying concept is
defined by a polygonal space partitioning, it is preferable to use
oblique decision trees for classification.  For example, there exist
many domains in which one or two oblique hyperplanes will be the best
model to use for classification.  In such domains, axis-parallel
methods will have to approximate the correct model with a
staircase-like structure, while an oblique tree-building method could
capture it with a tree that was both smaller and more
accurate.\footnote{Note that though a given oblique tree may have fewer
leaf nodes than an axis-parallel tree---which is what we mean by
``smaller''---the oblique tree may in some cases be larger in terms of
information content, because of the increased complexity of the tests
at each node.}  Figure~\ref{figure:stairs} gives an illustration.
\begin{figure}
\vspace{2.0in}
\special{psfile="stairs.ps" hoffset=10 voffset=-305}
\caption{The left side shows a simple 2-D domain in which two oblique 
hyperplanes define the classes.  The right side shows an approximation of
the sort that an axis-parallel decision tree would have to create to model
this domain.}
\label{figure:stairs}
\vspace*{-.2in} 
\end{figure}

Breiman et al.\ first suggested a method for inducing oblique decision
trees in 1984.  However, there has been very little further research
on such trees until relatively recently \cite{utgoff/brodley/90,%
heath/etal/93,murthy/etal/93,brodley/utgoff/94}.  A comparison of
existing approaches is given in more detail in Section
\ref{section:oblique}.  The purpose of this study is to review the
strengths and weaknesses of existing methods, to design a system that
combines some of the strengths and overcomes the weaknesses, and to
evaluate that system empirically and analytically.  The main
contributions and conclusions of our study are as follows:
\begin{itemize}
\item We have developed a new, randomized algorithm for inducing
oblique decision trees from examples.  This algorithm extends the
original 1984 work of Breiman et al.  Randomization helps
significantly in learning many concepts.

\item Our algorithm is fully implemented as an oblique decision tree 
induction system and is available over the Internet.  The code can be
retrieved from Online Appendix 1 of this paper (or by anonymous ftp
>from ftp://ftp.cs.jhu.edu/pub/oc1/oc1.tar.Z).

\item The randomized hill-climbing algorithm used in OC1 is more 
efficient than other existing randomized oblique decision tree methods
(described below).  In fact, the current implementation of OC1
guarantees a worst-case running time that is only $O(\log n)$ times
greater than the worst-case time for inducing axis-parallel trees
(i.e., $O(dn^2 \log n)$ vs.\ $O(dn^2)$).

\item The ability to generate oblique trees often produces very
small trees compared to axis-parallel methods.  When the underlying
problem requires an oblique split, oblique trees are also more
accurate than axis-parallel trees.  Allowing a tree-building system to
use both oblique and axis-parallel splits broadens the range of
domains for which the system should be useful.
\end{itemize}

The remaining sections of the paper follow this outline: the remainder
of this section briefly outlines the general paradigm of decision tree
induction, and discusses the complexity issues involved in inducing
oblique decision trees. Section~\ref{section:oblique} briefly reviews
some existing techniques for oblique DT induction, outlines some
limitations of each approach, and introduces the OC1 system.
Section~\ref{section:oc1} describes the OC1 system in detail.
Section~\ref{section:exp} describes experiments that (1)~compare the
performance of OC1 to that of several other axis-parallel and oblique
decision tree induction methods on a range of real-world datasets and
(2)~demonstrate empirically that OC1 significantly benefits from its
randomization methods.  In Section~\ref{section:conclude}, we conclude
with some discussion of open problems and directions for further
research.

\subsection{Top-Down Induction of Decision Trees}
Algorithms for inducing decision trees follow an approach
described by Quinlan as top-down induction of decision trees 
\citeyear{quinlan/86}.  This can also be called a greedy
divide-and-conquer method.  The basic outline is as follows:
\begin{enumerate}
\itemsep -0.0in
\item Begin with a set of examples called the training set, $T$.
If all examples in $T$ belong to one class, then halt.  
\item Consider all tests that divide $T$ into two or more subsets.
Score each test according to how well it splits up the examples.
\item Choose (``greedily'') the test that scores the highest.
\item Divide the examples into subsets and run this procedure 
recursively on each subset.
\end{enumerate}
Quinlan's original model only considered attributes with symbolic
values; in that model, a test at a node splits an attribute into all
of its values.  Thus a test on an attribute with three values will
have at most three child nodes, one corresponding to each value.  The
algorithm considers {\it all} possible tests and chooses
the one that optimizes a pre-defined goodness measure.  (One could
also split symbolic values into two or more subsets of values, which
gives many more choices for how to split the examples.)  As we explain
next, oblique decision tree methods cannot consider all tests due to
complexity considerations.

\subsection{Complexity of Induction of Oblique Decision Trees}

One reason for the relatively few papers on the problem of inducing
oblique decision trees is the increased computational complexity of
the problem when compared to the axis-parallel case.  There are two
important issues that must be addressed.  In the context of top-down
decision tree algorithms, we must address the complexity of finding
optimal separating hyperplanes (decision surfaces) for a given node of
a decision tree.  An optimal hyperplane will minimize the impurity
measure used; e.g., impurity might be measured by the total number of
examples mis-classified.  The second issue is the lower bound on the
complexity of finding optimal (e.g., smallest size) trees.

Let us first consider the issue of the complexity of selecting an
optimal oblique hyperplane for a single node of a tree.  In a domain
with $n$ training instances, each described using $d$ real-valued
attributes, there are at most $2^d \cdot {n \choose d}$ distinct
$d$-dimensional oblique splits; i.e., hyperplanes\footnote{Throughout
the paper, we use the terms ``split'' and ``hyperplane''
interchangeably to refer to the test at a node of a decision tree.
The first usage is standard~\cite{moret/82}, and refers to the fact
that the test splits the data into two partitions.  The second usage
refers to the geometric form of the test.} that divide the training
instances uniquely into two nonoverlapping subsets.  This upper bound
derives from the observation that every subset of size $d$ from the
$n$ points can define a $d$-dimensional hyperplane, and each such
hyperplane can be rotated slightly in $2^d$ directions to divide the 
set of $d$ points in all possible ways.  Figure \ref{figure:upperlimit}
illustrates these upper limits for two points in two dimensions.
\begin{figure}
\vspace{3.1in}
\special{psfile="upperlimit.ps" hoffset=160 voffset=-75 }
\caption{For $n$ points in $d$ dimensions ($n \geq d$), 
there are $n \cdot d$ distinct axis-parallel splits, while
there are $2^d \cdot {n \choose d}$ distinct
$d$-dimensional oblique splits.  This shows all distinct oblique
and axis-parallel splits for two specific points in 2-D.}
\label{figure:upperlimit}
\vspace*{-.2in} 
\end{figure}
For axis-parallel splits, there are only $n \cdot d$ distinct
possibilities, and axis-parallel methods such as C4.5
\cite{quinlan/93} and CART \cite{breiman/etal/84} can exhaustively 
search for the best split at each node.  The problem of searching for
the best oblique split is therefore much more difficult than that of
searching for the best axis-parallel split.  In fact, the problem is
NP-hard.

More precisely, Heath \citeyear{heath/92} proved that the following
problem is NP-hard: given a set of labelled examples, find the
hyperplane that minimizes the number of misclassified examples both
above and below the hyperplane.  This result implies that any method
for finding the optimal oblique split is likely to have exponential
cost (assuming $P \neq NP$).  Intuitively, the problem is that it is
impractical to enumerate all $2^d \cdot {n \choose d}$ distinct
hyperplanes and choose the best, as is done in axis-parallel decision
trees.  However, any non-exhaustive deterministic algorithm for
searching through all these hyperplanes is prone to getting stuck in
local minima.

On the other hand, it is possible to define impurity measures for
which the problem of finding optimal hyperplanes can be solved in
polynomial time.  For example, if one minimizes the sum of distances
of mis-classified examples, then the optimal solution can be found
using linear programming methods (if distance is measured along one
dimension only).  However, classifiers are usually judged by how many
points they classify correctly, regardless of how close to the
decision boundary a point may lie.  Thus most of the standard measures
for computing impurity base their calculation on the discrete number
of examples of each category on either side of the hyperplane.
Section~\ref{section:details} discusses several commonly used impurity
measures.

Now let us address the second issue, that of the complexity of
building a small tree.  It is easy to show that the problem of
inducing the smallest axis-parallel decision tree is NP-hard.  This
observation follows directly from the work of Hyafil and Rivest
\citeyear{hyafil/rivest/76}.  Note that one can generate the smallest
axis-parallel tree that is consistent with the training set in
polynomial time {\it if} the number of attributes is a constant.  This
can be done by using dynamic programming or branch and bound
techniques (see Moret \citeyear{moret/82} for several pointers).  But
when the tree uses oblique splits, it is not clear, even for a fixed
number of attributes, how to generate an optimal (e.g., smallest)
decision tree in polynomial time.  This suggests that the complexity
of constructing good oblique trees is greater than that for
axis-parallel trees.

It is also easy to see that the problem of constructing an optimal
(e.g., smallest) oblique decision tree is NP-hard.  This conclusion
follows from the work of Blum and Rivest \citeyear{blum/rivest/88}.
Their result implies that in $d$ dimensions (i.e., with $d$
attributes) the problem of producing a 3-node oblique decision tree
that is consistent with the training set is NP-complete.  More
specifically, they show that the following decision problem is
NP-complete: given a training set $T$ with $n$ examples and $d$
Boolean attributes, does there exist a 3-node neural network
consistent with $T$?  From this it is easy to show that the following
question is NP-complete: given a training set $T$, does there exist a
3-leaf-node oblique decision tree consistent with $T$?

As a result of these complexity considerations, we took the pragmatic
approach of trying to generate small trees, but not looking for the
smallest tree.  The greedy approach used by OC1 and virtually all
other decision tree algorithms implicitly tries to generate small
trees.  In addition, it is easy to construct example problems for
which the optimal split at a node will not lead to the best tree; thus
our philosophy as embodied in OC1 is to find locally good splits, but
not to spend excessive computational effort on improving the quality
of these splits.

\section{Previous Work on Oblique Decision Tree Induction}
\label{section:oblique}

Before describing the OC1 algorithm, we will briefly discuss some
existing oblique DT induction methods, including CART with linear
combinations, Linear Machine Decision Trees, and Simulated Annealing
of Decision Trees.  There are also methods that induce tree-like
classifiers with linear discriminants at each node, most notably
methods using linear programming 
\cite{mangasarian/etal/90,bennett/mangasarian/92,%
bennett/mangasarian/94a,bennett/mangasarian/94b}.  Though these
methods can find the optimal linear discriminants for specific
goodness measures, the size of the linear program grows very fast with
the number of instances and the number of attributes.  There is also
some less closely related work on algorithms to train artificial
neural networks to build decision tree-like classifiers
\cite{brent/91,cios/liu/92,herman/yeung/92}.

The first oblique decision tree algorithm to be proposed was CART with
linear combinations \cite[chapter 5]{breiman/etal/84}.  This
algorithm, referred to henceforth as CART-LC, is an important basis
for OC1. Figure ~\ref{figure:cart-lc} summarizes (using Breiman et al.'s
notation) what the CART-LC algorithm does at each node in the decision
tree.
\begin{figure}
{\small
{\bf
\begin{tabbing}
To \= induce a split at node $T$ of the decision tree:\\
     \> Normalize values for all $d$ attributes.\\
     \> $L = 0$\\
     \> While \= (TRUE)\\
     \>    \> $L = L+1$ \\
     \>    \> Let the current split $s_L$ be $v \leq c$, where \( v = \sum_{i=1}^{d} {a_ix_i} \).\\
     \>    \> For \= $i = 1,\dots,d$\\
     \>    \>     \> For \= $\gamma$ = -0.25,0,0.25 \\
     \>    \>     \>     \> Search for the $\delta$ that maximizes the goodness of the split $v - \delta(a_i + \gamma) \leq c$. \\
     \>    \>     \> Let $\delta^*$,$\gamma^*$ be the settings that result in highest goodness in these 3 searches.\\
     \>    \>     \> $a_i = a_i - \delta^*$, $c = c - \delta^*\gamma^*$.\\
     \>    \> Perturb $c$ to maximize the goodness of $s_L$, keeping $a_1,\ldots,a_d$ constant.\\
     \>    \> If $|$goodness($s_L$) - goodness$(s_{L-1})| \leq \epsilon$ exit while loop.\\
     \> Eliminate irrelevant attributes in \{$a_1,\ldots,a_d$\} using backward elimination.\\
     \> Convert $s_L$ to a split on the un-normalized attributes. \\
     \> Return the better of $s_L$ and the best axis-parallel split as 
the split for $T$.
\end{tabbing}
}
}
\caption{The procedure used by CART with linear combinations (CART-LC) at
each node of a decision tree.}
\label{figure:cart-lc}
\end{figure}
The core idea of the CART-LC algorithm is how it finds the value of
$\delta$ that maximizes the goodness of a split.  This idea is also used
in OC1, and is explained in detail in Section~\ref{section:perturb}.

After describing CART-LC, Breiman et al.\ point out that there is
still much room for further development of the algorithm.  OC1
represents an extension of CART-LC that includes some significant
additions.  It addresses the following limitations of CART-LC:
\begin{itemize}
\item CART-LC is fully deterministic. There is no built-in mechanism for
escaping local minima, although such minima may be very common for
some domains. Figure~\ref{figure:local-minimum} shows a simple
example for which CART-LC gets stuck.
\begin{figure}
\vspace{2.0in}
\special{psfile="small.ps" hscale=80 vscale=25 hoffset=-30 voffset=-10}
\caption{The deterministic perturbation algorithm of CART-LC fails to find 
the correct split for this data, even when it starts from the location of
the best axis-parallel split.  OC1 finds the correct split using one random 
jump.}
\label{figure:local-minimum}
\end{figure}
\item CART-LC produces only a single tree for a given data set.
\item CART-LC sometimes makes adjustments that increase the impurity
of a split.  This feature was probably included to allow it to escape
some local minima.
\item There is no upper bound on the time spent at any node in the
decision tree.  It halts when no perturbation changes the impurity
more than $\epsilon$, but because impurity may increase and decrease,
the algorithm can spend arbitrarily long time at a node.
\end{itemize}

Another oblique decision tree algorithm, one that uses a very
different approach from CART-LC, is the Linear Machine Decision Trees
(LMDT) system \cite{utgoff/brodley/91,brodley/utgoff/92}, which is a
successor to the Perceptron Tree method
\cite{utgoff/89b,utgoff/brodley/90}.  Each internal node in an LMDT 
tree is a Linear Machine \cite{nilsson/90}.  The training algorithm
presents examples repeatedly at each node until the linear machine
converges.  Because convergence cannot be guaranteed, LMDT uses
heuristics to determine when the node has stabilized.  To make the
training stable even when the set of training instances is not
linearly separable, a ``thermal training'' method \cite{frean/90} is
used, similar to simulated annealing.  

A third system that creates oblique trees is Simulated Annealing of
Decision Trees (SADT) \cite{heath/etal/93} which, like OC1, uses
randomization.  SADT uses simulated annealing
\cite{kirkpatrick/etal/83} to find good values for the coefficients of
the hyperplane at each node of a tree.  SADT first places a hyperplane
in a canonical location, and then iteratively perturbs all the
coefficients by small random amounts.  Initially, when the temperature
parameter is high, SADT accepts almost any perturbation of the
hyperplane, regardless of how it changes the goodness score.  However,
as the system ``cools down,'' only changes that improve the goodness
of the split are likely to be accepted.  Though SADT's use of
randomization allows it to effectively avoid some local minima, it
compromises on efficiency.  It runs much slower than either CART-LC,
LMDT or OC1, sometimes considering tens of thousands of hyperplanes
at a single node before it finishes annealing.

Our experiments in Section \ref{section:exp2} include some results
showing how all of these methods perform on three artificial domains.

We next describe a way to combine some of the strengths of the methods
just mentioned, while avoiding some of the problems.  Our algorithm,
OC1, uses deterministic hill climbing most of the time, ensuring
computational efficiency.  In addition, it uses two kinds of
randomization to avoid local minima.  By limiting the number of random
choices, the algorithm is guaranteed to spend only polynomial time at
each node in the tree.  In addition, randomization itself has produced
several benefits: for example, it means that the algorithm can produce
many different trees for the same data set.  This offers the
possibility of a new family of classifiers: $k$-decision-tree
algorithms, in which an example is classified by the majority vote of
$k$ trees.  Heath et al.~\citeyear{heath/etal/93b} have shown that
$k$-decision tree methods (which they call $k$-DT) will consistently
outperform single tree methods if classification accuracy is the main
criterion.  Finally, our experiments indicate that OC1 efficiently
finds small, accurate decision trees for many different types of
classification problems.

\section{Oblique Classifier 1 (OC1)}
\label{section:oc1}
In this section we discuss details of the oblique decision tree
induction system OC1.  As part of this description, we include:
\begin{itemize}
\itemsep -0.0in
\item the method for finding coefficients of a hyperplane at each 
tree node,
\item methods for computing the impurity or goodness of a hyperplane,
\item a tree pruning strategy, and
\item methods for coping with missing and irrelevant attributes.
\end{itemize} 
Section~\ref{section:perturb} focuses on the most complicated of
these algorithmic details; i.e. the question of how to find a
hyperplane that splits a given set of instances into two reasonably
``pure'' non-overlapping subsets.  This randomized perturbation
algorithm is the main novel contribution of OC1\@.
Figure~\ref{figure:oc1-alg} summarizes the basic OC1 algorithm, used
at each node of a decision tree.
\begin{figure}
{\small
{\bf
\begin{tabbing}
abc \= abc \= abc \= \kill
To find a split of a set of examples $T$: \\
   \> Find the best axis-parallel split of $T$.  Let $I$ be the
      impurity of this split. \\
   \> Repeat $R$ times: \\
   \> \> Choose a random hyperplane $H$. \\
   \> \> (For the first iteration, initialize $H$ to be the best 
axis-parallel split.) \\
   \> \> Step 1: Until the impurity measure does not improve, do: \\
   \> \> \> Perturb each of the coefficients of $H$ in sequence. \\
   \> \> Step 2: Repeat at most $J$ times: \\
   \> \> \> Choose a random direction and attempt to perturb $H$ in 
            that direction. \\
   \> \> \> If this reduces the impurity of $H$, go to Step 1. \\
   \> \> Let $I_1$ = the impurity of $H$.  If $I_1 < I$, then set $I=I_1$. \\
   \> Output the split corresponding to $I$.
\end{tabbing}
}
}
\vspace*{-.2in} 
\caption{Overview of the OC1 algorithm for a single node of a decision tree.}
\label{figure:oc1-alg}
\end{figure}
This figure will be explained further in the following sections.

\subsection{Perturbation algorithm}
\label{section:perturb}
OC1 imposes no restrictions on the orientation of the hyperplanes.
However, in order to be at least as powerful as standard DT methods,
it first finds the best axis-parallel (univariate) split at a node
before looking for an oblique split.  OC1 uses an oblique split only
when it improves over the best axis-parallel split.\footnote{As
pointed out in \cite[Chapter 5]{breiman/etal/84}, it does not make
sense to use an oblique split when the number of examples at a node
$n$ is less than or almost equal to the number of features $d$,
because the data {\it underfits} the concept. By default, OC1 uses
only axis-parallel splits at tree nodes at which $n < 2d$. The user
can vary this threshold.}

The search strategy for the space of possible hyperplanes is defined
by the procedure that perturbs the current hyperplane $H$ to a new
location.  Because there are an exponential number of distinct ways to
partition the examples with a hyperplane, any procedure that simply
enumerates all of them will be unreasonably costly.  The two main
alternatives considered in the past have been simulated annealing,
used in the SADT system \cite{heath/etal/93}, and deterministic
heuristic search, as in CART-LC \cite{breiman/etal/84}.  OC1 combines
these two ideas, using heuristic search until it finds a local
minimum, and then using a non-deterministic search step to get out of
the local minimum.  (The non-deterministic step in OC1 is {\it not}
simulated annealing, however.)

We will start by explaining how we perturb a hyperplane to split the
training set $T$ at a node of the decision tree.  Let $n$ be the
number of examples in $T$, $d$ be the number of attributes (or
dimensions) for each example, and $k$ be the number of categories.
Then we can write $T_j = (x_{j1},x_{j2},\ldots,x_{jd},C_j)$ for the
$j$th example from the training set $T$, where $x_{ji}$ is the value
of attribute $i$ and $C_j$ is the category label.  As defined
in Eq.~\ref{equation:hyperplane}, the 
equation of the current hyperplane $H$ at a node of the decision tree
is written as \( \sum_{i=1}^{d} (a_i x_i) + a_{d+1} = 0 \). 
If we substitute a point (an example) $T_j$ into the equation for $H$,
we get \( \sum_{i=1}^{d} (a_i x_{ji}) + a_{d+1}  = V_j\),
where the sign of $V_j$ tells us whether the point $T_j$ is above or
below the hyperplane $H$; i.e., if $V_j > 0$, then $T_j$ is above $H$.
If $H$ splits the training set $T$ perfectly, then all points
belonging to the same category will have the same sign for
$V_j$. i.e., sign($V_i$) = sign($V_j$) iff category($T_i$) =
category($T_j$).

OC1 adjusts the coefficients of $H$ individually, finding a locally
optimal value for one coefficient at a time.  This key idea was
introduced by Breiman et al.  It works as follows.  Treat the
coefficient $a_m$ as a variable, and treat all other coefficients as
constants.  Then $V_j$ can be viewed as a function of $a_m$. 
In particular, the condition that $T_j$ is above $H$ is equivalent
to 
\[   V_j > 0  \]
\begin{equation}
\label{equation:uj}
   a_m > \frac{a_m x_{jm} - V_j}{x_{jm}} \stackrel{\rm def}{=} U_j 
\end{equation}
assuming that $x_{jm} > 0$, which we ensure by normalization.  Using
this definition of $U_j$, the point $T_j$ is above $H$ if $a_m > U_j$,
and below otherwise.  By plugging all the points from $T$ into this
equation, we will obtain $n$ constraints on the value of $a_m$.

The problem then is to find a value for $a_m$ that satisfies as many of
these constraints as possible.  (If all the constraints are satisfied,
then we have a perfect split.)  This problem is easy to solve optimally:
simply sort all the values $U_j$, and consider setting $a_m$ to
the midpoint between each pair of different values.  This is illustrated in
Figure~\ref{figure:1Dsplit}.
\begin{figure}
\vspace{1.0in}
 \special{psfile="1Dsplit.ps" hoffset=-20 voffset=-320}
\caption{Finding the optimal value for a single coefficient $a_m$.  Large
U's correspond to examples in one category and small u's to another.}
\label{figure:1Dsplit}
\vspace*{-.2in} 
\end{figure}
In the figure, the categories are indicated by font size; the larger
$U_i$'s belong to one category, and the smaller to another.  For each
distinct placement of the coefficient $a_m$, OC1 computes the impurity
of the resulting split; e.g., for the location between $U_6$ and $U_7$
illustrated here, two examples on the left and one example on the
right would be misclassified (see Section~\ref{section:impmeasures}
for different ways of computing impurity).  As the figure illustrates,
the problem is simply to find the best one-dimensional split of the
$U$s, which requires considering just $n-1$ values for $a_m$.  The
value $a_m'$ obtained by solving this one-dimensional problem is
then considered as a replacement for $a_m$. Let $H_1$ be the
hyperplane obtained by ``perturbing'' $a_m$ to $a_m'$.  If $H$ has
better (lower) impurity than $H_1$, then $H_1$ is discarded. If $H_1$
has lower impurity, $H_1$ becomes the new location of the hyperplane.
If $H$ and $H_1$ have identical impurities, then $H_1$ replaces $H$
with probability $P_{stag}$.\footnote{The parameter $P_{stag}$,
denoting ``stagnation probability'', is the probability that a
hyperplane is perturbed to a location that does not change the
impurity measure.  To prevent the impurity from remaining stagnant for
a long time, $P_{stag}$ decreases by a constant amount each time OC1
makes a ``stagnant'' perturbation; thus only a constant number of
such perturbations will occur at each node.  This constant can be set
by the user.  $P_{stag}$ is reset to 1 every time the global impurity
measure is improved.} Figure~\ref{figure:perturb} contains pseudocode
for our perturbation procedure.
\begin{figure}
{\small
{\bf
\begin{tabbing}
\noindent Perturb(H,m) \\
\{  \= \kill
    \> For \= $j = 1,\ldots,n$\\
    \>     \> Compute $U_j$ (Eq.~\ref{equation:uj})\\ 
    \> Sort $U_1,\ldots,U_n$ in non-decreasing order.\\
    \> $a_m'$ = best univariate split of the sorted $U_j$s.\\
    \> $H_1$ = result of substituting $a_m'$ for $a_m$ in $H$.\\
    \> If \((impurity(H_1) < impurity(H)) \)\\
    \>    \> \{ $a_m = a_m'$ ; $P_{move} = P_{stag}$  \} \\
    \> Else if \((impurity(H) = impurity(H_1)) \) \\
    \>  \> \{  \= $a_m = a_m'$ \=with probability $P_{move}$\\
    \>  \>     \> $P_{move} = P_{move} - 0.1 * P_{stag}$ \} \\
\end{tabbing}
}
}
\vspace*{-.3in}
\caption{Perturbation algorithm for a single coefficient $a_m$.}
\label{figure:perturb}
\vspace*{-.1in}
\end{figure}

Now that we have a method for locally improving a coefficient of a
hyperplane, we need to decide which of the $d+1$ coefficients to pick
for perturbation.  We experimented with three different methods for
choosing which coefficient to adjust, namely, sequential, best first
and random.
\begin{tabbing}
{\bf Seq}: \hspace*{0.1in} \= Rep\=eat until none of the coefficient values is 
              modified in the {\bf For} loop:\\
             \>        \>For $i=1$ to $d$, Perturb($H,i$)\\
{\bf Best}:  \> Repeat until coefficient $m$ remains unmodified: \\
           \>        \>$m$ = \=coefficient which when perturbed, results in the \\
           \>        \>    \>maximum improvement of the impurity measure.\\
           \>        \>Perturb($H,m$)\\
{\bf R-50}:  \> Repeat a fixed number of times (50 in our experiments): \\
           \>        \>$m$ = random integer between 1 and $d+1$\\
           \>        \>Perturb($H,m$)\\
\end{tabbing}     
Our previous experiments \cite{murthy/etal/93} indicated that the
order of perturbation of the coefficients does not affect the
classification accuracy as much as other parameters, especially the
randomization parameters (see below).  Since none of these orders was
uniformly better than any other, we used sequential (Seq) perturbation
for all the experiments reported in Section~\ref{section:exp}.
 

\subsection{Randomization}
\label{section:rand}
The perturbation algorithm halts when the split reaches a local
minimum of the impurity measure.  For OC1's search space, a local
minimum occurs when no perturbation of any single coefficient of the
current hyperplane will decrease the impurity measure.  (Of course, a
local minimum may also be a global minimum.)  We have implemented two
ways of attempting to escape local minima: perturbing the hyperplane
with a random vector, and re-starting the perturbation algorithm with
a different random initial hyperplane.

The technique of perturbing the hyperplane with a random vector works
as follows.  When the system reaches a local minimum, it chooses a
random vector to add to the coefficients of the current hyperplane.
It then computes the optimal amount by which the hyperplane should be
perturbed along this random direction.  To be more precise, when a
hyperplane \(H = \sum_{i=1}^{d} a_i x_i + a_{d+1} \) cannot be
improved by deterministic perturbation, OC1 repeats the following loop
$J$ times (where $J$ is a user-specified parameter, set to 5 by
default).
\begin{itemize}
\itemsep 0.0in
\item Choose a random vector $R = (r_1,r_2,\ldots,r_{d+1})$.
\item Let $\alpha$ be the amount by which we want to perturb $H$ in the
direction $R$.  In other words, let
\(H_1 = \sum_{i=1}^{d} {(a_i + \alpha r_i) x_i} + (a_{d+1} + 
    \alpha r_{d+1}) \).
\item Find the optimal value for $\alpha$.
\item If the hyperplane $H_1$ thus obtained decreases the overall impurity, 
replace $H$ with $H_1$, exit this loop and begin the deterministic
perturbation algorithm for the individual coefficients.
\end{itemize}
Note that we can treat $\alpha$ as the only variable in the equation
for $H_1$.  Therefore each of the $n$ examples in $T$, if plugged into
the equation for $H_1$, imposes a constraint on the value of $\alpha$.
OC1 therefore can use its coefficient perturbation method (see
Section~\ref{section:perturb}) to compute the best value of $\alpha$.
If $J$ random jumps fail to improve the impurity, OC1 halts and uses
$H$ as the split for the current tree node.

An intuitive way of understanding this random jump is to look at the
dual space in which the algorithm is actually searching.  Note that
the equation \(H = \sum_{i=1}^{d} a_i x_i + a_{d+1} \) defines a space
in which the axes are the coefficients $a_i$ rather than the
attributes $x_i$.  Every point in this space defines a distinct
hyperplane in the original formulation.  The deterministic algorithm
used in OC1 picks a hyperplane and then adjusts coefficients one at a
time.  Thus in the dual space, OC1 chooses a point and perturbs it by
moving it parallel to the axes.  The random vector $R$ represents a
random {\it direction} in this space.  By finding the best value for
$\alpha$, OC1 finds the best distance to adjust the hyperplane in the
direction of $R$.

Note that this additional perturbation in a random direction does not
significantly increase the time complexity of the algorithm (see
Appendix A).  We found in our experiments that even a single random
jump, when used at a local minimum, proves to be very helpful.
Classification accuracy improved for every one of our data sets when
such perturbations were made.  See Section \ref{section:exp2} for some
examples.

The second technique for avoiding local minima is a variation on the
idea of performing multiple local searches. The technique of multiple
local searches is a natural extension to local search, and has been
widely mentioned in the optimization literature (see Roth
\citeyear{roth/70} for an early example).  Because most of the steps
of our perturbation algorithm are deterministic, the initial
hyperplane largely determines which local minimum will be encountered
first.  Perturbing a single initial hyperplane is thus unlikely to
lead to the best split of a given data set.  In cases where the random
perturbation method fails to escape from local minima, it may be
helpful to simply start afresh with a new initial hyperplane.  We use
the word {\it restart} to denote one run of the perturbation
algorithms, at one node of the decision tree, using one random initial
hyperplane.\footnote{The first run through the algorithm at each node
always begins at the location of the best axis-parallel hyperplane;
all subsequent restarts begin at random locations.}  That is, a
restart cycles through and perturbs the coefficients one at a time and
then tries to perturb the hyperplane in a random direction when the
algorithm reaches a local minimum.  If this last perturbation reduces
the impurity, the algorithm goes back to perturbing the coefficients
one at a time.  The restart ends when neither the deterministic local
search nor the random jump can find a better split.  One of the
optional parameters to OC1 specifies how many restarts to use.  If
more than one restart is used, then the best hyperplane found thus far
is always saved.  In all our experiments, the classification
accuracies increased with more than one restart.  Accuracy tended to
increase up to a point and then level off (after about 20--50
restarts, depending on the domain).  Overall, the use of multiple
initial hyperplanes substantially improved the quality of the decision
trees found (see Section~\ref{section:exp2} for some examples).
 
By carefully combining hill-climbing and randomization, OC1 ensures a
worst case time of $O(dn^2 \log n)$ for inducing a decision tree.  See
Appendix A for a derivation of this upper bound.

\paragraph{Best Axis-Parallel Split.}
It is clear that axis-parallel splits are more suitable for some data
distributions than oblique splits. To take into account such
distributions, OC1 computes the best axis-parallel split {\em and} an
oblique split at each node, and then picks the better of the
two.\footnote{Sometimes a simple axis-parallel split is preferable to
an oblique split, even if the oblique split has slightly lower
impurity. The user can specify such a bias as an input parameter to
OC1.} Calculating the best axis-parallel split takes an additional
$O(dn \log n)$ time, and so does not increase the asymptotic time
complexity of OC1.  As a simple variant of the OC1 system, the user
can opt to ``switch off'' the oblique perturbations, thus building an
axis-parallel tree on the training data.  Section~\ref{section:exp1}
empirically demonstrates that this axis-parallel variant of OC1
compares favorably with existing axis-parallel algorithms.

\subsection{Other Details}
\label{section:details}

\subsubsection{Impurity Measures}
\label{section:impmeasures}
OC1 attempts to divide the $d$-dimensional attribute space into
homogeneous regions; i.e., regions that contain examples from just one
category.  The goal of adding new nodes to a tree is to split up the
sample space so as to minimize the ``impurity'' of the training set.
Some algorithms measure ``goodness'' instead of impurity, the
difference being that goodness values should be maximized while
impurity should be minimized.  Many different measures of
impurity have been studied \cite{breiman/etal/84,quinlan/86,%
mingers/89a,buntine/niblett/92,fayyad/irani/92b,heath/etal/93}. 

The OC1 system is designed to work with a large class of impurity
measures.  Stated simply, if the impurity measure uses only the counts
of examples belonging to every category on both sides of a split, then
OC1 can use it.  (See Murthy and Salzberg
\citeyear{murthy/salzberg/94} for ways of mapping other kinds of
impurity measures to this class of impurity measures.)  The user can
plug in any impurity measure that fits this description.  The OC1
implementation includes six impurity measures, namely:
\begin{enumerate}
\itemsep -0.1in
\item Information Gain
\item The Gini Index
\item The Twoing Rule
\item Max Minority
\item Sum Minority
\item Sum of Variances
\end{enumerate}  
Though all six of the measures have been defined elsewhere in the
literature, in some cases we have made slight modifications that are
defined precisely in Appendix B\@.  Our experiments
indicated that, on average, Information Gain, Gini Index and the
Twoing Rule perform better than the other three measures for both
axis-parallel and oblique trees.  The Twoing Rule is the current
default impurity measure for OC1, and it was used in all of the
experiments reported in Section~\ref{section:exp}.  There are,
however, artificial data sets for which Sum Minority and/or Max
Minority perform much better than the rest of the measures.  For
instance, Sum Minority easily induces the exact tree for the POL data
set described in Section~\ref{section:artificialdata}, while all other
methods have difficulty finding the best tree.

\paragraph{Twoing Rule.}
The Twoing Rule was first proposed by Breiman et al.\ (1984).  The
value to be computed is defined as:
\[ \mbox{TwoingValue} = (|T_L|/n) * (|T_R|/n) * 
          (\sum_{i=1}^{k} {\left| L_i/|T_L| - R_i/|T_R| \right|})^2 \]
where $|T_L|$ ($|T_R|$) is the number of examples on the left (right)
of a split at node $T$, $n$ is the number of examples at node $T$, and
$L_i$ ($R_i$) is the number of examples in category $i$ on the left
(right) of the split.  The TwoingValue is actually a goodness measure
rather than an impurity measure.  Therefore OC1 attempts to minimize
the reciprocal of this value.

The remaining five impurity measures implemented in OC1 are defined in
Appendix B.

\subsubsection{Pruning}
\label{section:pruning}
Virtually all decision tree induction systems prune the trees they
create in order to avoid overfitting the data.  Many studies have
found that judicious pruning results in both smaller and more accurate
classifiers, for decision trees as well as other types of machine
learning systems \cite{quinlan/87b,niblett/86,cestnik/etal/87,%
kodratoff/manago/87,cohen/93,hassibi/stork/93,wolpert/92,schaffer/93}.
For the OC1 system we implemented an existing pruning method, but note
that any tree pruning method will work fine within OC1.  Based on the
experimental evaluations of Mingers \citeyear{mingers/89b} and other
work cited above, we chose Breiman et al.'s Cost Complexity (CC)
pruning \citeyear{breiman/etal/84} as the default pruning method for OC1.
This method, which is also called Error Complexity or Weakest Link
pruning, requires a separate pruning set.  The pruning set can be a
randomly chosen subset of the training set, or it can be approximated
using cross validation.  OC1 randomly chooses 10\% (the default value)
of the training data to use for pruning.  In the experiments reported
below, we only used this default value.

Briefly, the idea behind CC pruning is to create a set of trees of
decreasing size from the original, complete tree.  All these trees are
used to classify the pruning set, and accuracy is estimated from that.
CC pruning then chooses the smallest tree whose accuracy is within $k$
standard errors squared of the best accuracy obtained.  When the 0-SE
rule ($k=0$) is used, the tree with highest accuracy on the pruning
set is selected. When $k>0$, smaller tree size is preferred over
higher accuracy.  For details of Cost Complexity pruning, see Breiman
et al.~\citeyear{breiman/etal/84} or Mingers~\citeyear{mingers/89b}.


\subsubsection{Irrelevant attributes}

Irrelevant attributes pose a significant problem for most machine
learning methods
\cite{breiman/etal/84,aha/90,almuallin/dietterich/91,%
kira/rendell/92,salzberg/92,cardie/93,schlimmer/93,langley/sage/93,%
brodley/utgoff/94}.  Decision tree algorithms, even axis-parallel
ones, can be confused by too many irrelevant attributes.  Because
oblique decision trees learn the coefficients of each attribute at a
DT node, one might hope that the values chosen for each coefficient
would reflect the relative importance of the corresponding attributes.
Clearly, though, the process of searching for good coefficient values
will be much more efficient when there are fewer attributes; the
search space is much smaller.  For this reason, oblique DT induction
methods can benefit substantially by using a feature selection method
(an algorithm that selects a subset of the original attribute set) in
conjunction with the coefficient learning algorithm
\cite{breiman/etal/84,brodley/utgoff/94}.

Currently, OC1 does not have a built-in mechanism to select relevant
attributes.  However, it is easy to include any of several standard
methods (e.g., stepwise forward selection or stepwise backward
selection) or even an ad hoc method to select features before running
the tree-building process.  For example, in separate experiments on
data from the Hubble Space Telescope \cite{salzberg/etal/94}, we used
feature selection methods as a preprocessing step to OC1, and reduced
the number of attributes from 20 to 2.  The resulting decision trees
were both simpler and more accurate.  Work is currently underway to
incorporate an efficient feature selection technique into the OC1
system.  

Regarding missing values, if an example is missing a value for any
attribute, OC1 uses the mean value for that attribute.  One can of
course use other techniques for handling missing values, but those
were not considered in this study.

\section{Experiments}
\label{section:exp}

In this section, we present two sets of experiments to support the
following two claims.
\begin{enumerate}
\item OC1 compares favorably over a variety of real-world domains
with several existing axis-parallel and oblique decision tree induction
methods.
\item Randomization, both in the form of multiple local searches and
random jumps, improves the quality of decision trees produced by OC1.
\end{enumerate}

The experimental method used for all the experiments is described
in Section \ref{section:expmethod}.  Sections \ref{section:exp1} and
\ref{section:exp2} describe experiments corresponding to the
above two claims.  Each experimental section begins with a description
of the data sets, and then presents the experimental results and
discussion.

\subsection{Experimental Method}
\label{section:expmethod}
We used five-fold cross validation (CV) in all our experiments to
estimate classification accuracy.  A $k$-fold CV experiment consists
of the following steps.
\begin{enumerate}
\itemsep -0.1in
\item Randomly divide the data into $k$ equal-sized disjoint partitions.
\item For each partition, build a decision tree using all data outside the
partition, and test the tree on the data in the partition.
\item Sum the number of correct classifications of the $k$ trees and divide
by the total number of instances to compute the classification accuracy. 
Report this accuracy and the average size of the $k$ trees.
\end{enumerate}
Each entry in Tables \ref{table:1} and \ref{table:2} is a result of
ten 5-fold CV experiments; i.e., the result of tests that used 50
decision trees.  Each of the ten 5-fold cross validations used a
different random partitioning of the data.  Each entry in the tables
reports the mean and standard deviation of the classification
accuracy, followed by the mean and standard deviation of the decision
tree size (measured as the number of leaf nodes).  Good results should
have high values for accuracy, low values for tree size, and small
standard deviations.

In addition to OC1, we also included in the experiments an
axis-parallel version of OC1, which only considers axis-parallel
hyperplanes.  We call this version, described in
Section~\ref{section:rand}, OC1-AP\@.  In all our experiments, both
OC1 and OC1-AP used the Twoing Rule (Section \ref{section:impmeasures})
to measure impurity.  Other parameters to OC1 took their default
values unless stated otherwise.  (Defaults include the following:
number of restarts at each node: 20.  Number of random jumps attempted
at each local minimum: 5.  Order of coefficient perturbation:
Sequential.  Pruning method: Cost Complexity with the 0-SE rule, using
10\% of the training set exclusively for pruning.)

In our comparison, we used the oblique version of the CART algorithm,
CART-LC\@.  We implemented our own version of CART-LC, following the
description in Brei\-man et al.~\citeyear[Chapter 5]{breiman/etal/84};
however, there may be differences between our version and other
versions of this system (note that CART-LC is not freely available).
Our implementation of CART-LC measured impurity with the Twoing Rule
and used 0-SE Cost Complexity pruning with a separate test set, just as
OC1 does.  We did not include any feature selection methods in CART-LC
or in OC1, and we did not implement normalization.  Because the CART
coefficient perturbation algorithm may alternate indefinitely between
two locations of a hyperplane (see Section~\ref{section:oblique}), we
imposed an arbitrary limit of 100 such perturbations before forcing
the perturbation algorithm to halt.

We also included axis-parallel CART and C4.5 in our comparisons.  We
used the implementations of these algorithms from the IND 2.1
package~\cite{buntine/92}.  The default cart0 and c4.5 ``styles''
defined in the package were used, without altering any parameter
settings.  The cart0 style uses the Twoing Rule and 0-SE cost
complexity pruning with 10-fold cross validation.  The pruning method,
impurity measure and other defaults of the c4.5 style are the same as
those described in Quinlan \citeyear{quinlan/93}.

\subsection{OC1 vs. Other Decision Tree Induction Methods}
\label{section:exp1}
Table~\ref{table:1} compares the performance of OC1 to three
well-known decision tree induction methods plus OC1-AP on six
different real-world data sets.  In the next section we will consider
artificial data, for which the concept definition can be precisely
characterized.

\subsubsection{Description of Data Sets}
\label{section:realdata}

\paragraph{Star/Galaxy Discrimination.}
Two of our data sets came from a large set of astronomical images
collected by Odewahn et al. \cite{odewahn/etal/92}.  In their study,
they used these images to train artificial neural networks running the
perceptron and back propagation algorithms.  The goal was to
classify each example as either ``star'' or ``galaxy.''  Each image is
characterized by 14 real-valued attributes, where the attributes were
measurements defined by astronomers as likely to be relevant for this
task.  The objects in the image were divided by Odewahn et al. into
``bright'' and ``dim'' data sets based on the image intensity values,
where the dim images are inherently more difficult to classify.  (Note
that the ``bright'' objects are only bright in relation to others in
this data set.  In actuality they are extremely faint, visible only to
the most powerful telescopes.)  The bright set contains 2462 objects
and the dim set contains 4192 objects.

In addition to the results reported in Table~\ref{table:1}, the
following results have appeared on the Star/Galaxy data.  Odewahn et
al.~(1992) reported accuracy of 99.8\% accuracy on the bright objects,
and 92.0\% on the dim ones, although it should be noted that this
study used a single training and test set partition.  Heath
\citeyear{heath/92} reported 99.0\% accuracy on the bright objects using
SADT, with an average tree size of 7.03 leaves.  This study also used
a single training and test set.  Salzberg \citeyear{salzberg/92} reported
accuracies of 98.8\% on the bright objects, and 95.1\% on the dim
objects, using 1-Nearest Neighbor (1-NN) coupled with a feature
selection method that reduces the number of features.

\paragraph{Breast Cancer Diagnosis.} 
Mangasarian and Bennett have compiled data on the problem of
diagnosing breast cancer to test several new classification methods
\cite{mangasarian/etal/90,bennett/mangasarian/92,bennett/mangasarian/94a}.
This data represents a set of patients with breast cancer, where each
patient was characterized by nine numeric attributes plus the
diagnosis of the tumor as benign or malignant.  The data set currently
has 683 entries and is available from the UC Irvine machine learning
repository \cite{mlrepository}.  Heath et al.~\citeyear{heath/etal/93}
reported 94.9\% accuracy on a subset of this data set (it then had
only 470 instances), with an average decision tree size of 4.6 nodes,
using SADT\@.  Salzberg \citeyear{salzberg/91} reported 96.0\%
accuracy using 1-NN on the same (smaller) data set. Herman and Yeung
\citeyear{herman/yeung/92} reported 99.0\% accuracy using piece-wise
linear classification, again using a somewhat smaller data set.

\paragraph{Classifying Irises.}
This is Fisher's famous iris data, which has been extensively studied
in the statistics and machine learning literature. 
The data consists of 150 examples, where each example is described by
four numeric attributes.  There are 50 examples of each of three
different types of iris flower.  Weiss and Kapouleas
\citeyear{weiss/kapouleas/89} obtained accuracies of 96.7\% and 96.0\% on
this data with back propagation and 1-NN, respectively.

\paragraph{Housing Costs in Boston.}
This data set, also available as a part of the UCI ML repository,
describes housing values in the suburbs of Boston as a function of 12
continuous attributes and 1 binary attribute
\cite{harrison/rubinfeld/78}.  The category variable (median value of
owner-occupied homes) is actually continuous, but we discretized it so
that category = 1 if value $<$ \$21000, and 2 otherwise.  For other
uses of this data, see \cite{belsley/80,quinlan/93a}.

\paragraph{Diabetes diagnosis.}
This data catalogs the presence or absence of diabetes among Pima
Indian females, 21 years or older, as a function of eight
numeric-valued attributes.  The original source of the data is the
National Institute of Diabetes and Digestive and Kidney Diseases, and
it is now available in the UCI repository.  Smith et
al.~\citeyear{smith/etal/88} reported 76\% accuracy on this data using
their ADAP learning algorithm, using a different experimental method
>from that used here.

\begin{table}
\begin{center}
\begin{tabular}{|l|r|r|r|r|r|r|} \hline \hline
{\em Algorithm} &Bright S/G & Dim S/G & Cancer & Iris  & 
                 Housing    & Diabetes   \\ \hline 
OC1     & {\bf 98.9}$\pm$0.2 & {\bf 95.0}$\pm$0.3 & {\bf 96.2}$\pm$0.3 
          & 94.7$\pm$3.1 & 82.4$\pm$0.8 & {\bf 74.4}$\pm$1.0 \\
~       &  4.3$\pm$1.0 & 13.0$\pm$8.7 &  2.8$\pm$0.9 & 3.1$\pm$0.2 & 
                 6.9$\pm$3.2 & 5.4$\pm$3.8 \\ \hline
CART-LC & 98.8$\pm$0.2 & 92.8$\pm$0.5 & 95.3$\pm$0.6 & 93.5$\pm$2.9 &
                 81.4$\pm$1.2 & 73.7$\pm$1.2 \\
~       &  3.9$\pm$1.3 & 24.2$\pm$8.7 &  3.5$\pm$0.9 & 3.2$\pm$0.3 & 
                  5.8$\pm$3.2 & 8.0$\pm$5.2 \\ \hline
OC1-AP  & 98.1$\pm$0.2 & 94.0$\pm$0.2 & 94.5$\pm$0.5 & 92.7$\pm$2.4 &
                 81.8$\pm$1.0 & 73.8$\pm$1.0 \\
~       &  6.9$\pm$2.4 & 29.3$\pm$8.8 &  6.4$\pm$1.7 & 3.2$\pm$0.3 & 
                 8.6$\pm$4.5 &11.4$\pm$7.5 \\ \hline
CART-AP & 98.5$\pm$0.5 & 94.2$\pm$0.7 & 95.0$\pm$1.6 & 93.8$\pm$3.7 &
                 82.1$\pm$3.5 & 73.9$\pm$3.4 \\ 
~       & 13.9$\pm$5.7 & 30.4$\pm$10  & 11.5$\pm$7.2 & 4.3$\pm$1.6 &
                 15.1$\pm$10 & 11.5$\pm$9.1 \\ \hline
C4.5    & 98.5$\pm$0.5 & 93.3$\pm$0.8 & 95.3$\pm$2.0 & {\bf 95.1}$\pm$3.2 &
                 {\bf 83.2}$\pm$3.1 &71.4$\pm$3.3 \\
~       & 14.3$\pm$2.2 & 77.9$\pm$7.4 &  9.8$\pm$2.2 & 4.6$\pm$0.8 &
                 28.2$\pm$3.3 & 56.3$\pm$7.9 \\ \hline \hline
\end{tabular}
\caption{Comparison of OC1 and other decision tree induction methods on
six different data sets.  The first line for each method gives accuracies,
and the second line gives average tree sizes.  The highest accuracy for
each domain appears in boldface.}
\label{table:1}
\vspace*{-.2in}
\end{center}
\end{table}

\subsubsection{Discussion}
The table shows that, for the six data sets considered here, OC1
consistently finds better trees than the original oblique CART method.
Its accuracy was greater in all six domains, although the difference
was significant (more than 2 standard deviations) only for the dim
star/galaxy problem.  The average tree sizes were roughly equal for
five of the six domains, and for the dim stars and galaxies, OC1 found
considerably smaller trees.  These differences will be analyzed and
quantified further by using artificial data, in the following section.

Out of the five decision tree induction methods, OC1 has the highest
accuracy on four of the six domains: bright stars, dim stars, cancer
diagnosis, and diabetes diagnosis.  On the remaining two domains, OC1
has the second highest accuracy in each case.  Not surprisingly, the
oblique methods (OC1 and CART-LC) generally find much smaller trees
than the axis-parallel methods.  This difference can be quite striking
for some domains---note, for example, that OC1 produced a tree with
just 13 nodes on average for the dim star/galaxy problem, while C4.5
produced a tree with 78 nodes, 6 times larger.  Of course, in domains
for which an axis-parallel tree is the appropriate representation,
axis-parallel methods should compare well with oblique methods in
terms of tree size.  In fact, for the Iris data, all the methods found
similar-sized trees.

\subsection{Randomization Helps OC1}
\label{section:exp2}
In our second set of experiments, we examine more closely the effect
of introducing randomized steps into the algorithm for finding oblique
splits.  Our experiments demonstrate that OC1's ability to produce an
accurate tree from a set of training data is clearly enhanced by the
two kinds of randomization it uses.  More precisely, we use three
artificial data sets (for which the underlying concept is known to the
experimenters) to show that OC1's performance improves substantially
when the deterministic hill climbing is augmented in any of three
ways:
\begin{itemize}
\itemsep 0.0in
\item with multiple restarts from random initial locations, 
\item with perturbations in random directions at local minima, or
\item with both of the above randomization steps. 
\end{itemize}

In order to find clear differences between algorithms, one needs to
know that the concept underlying the data is indeed difficult to
learn.  For simple concepts (say, two linearly separable classes in
2-D), many different learning algorithms will produce very accurate
classifiers, and therefore the advantages of randomization may not be
detectable.  It is known that many of the commonly-used data sets from
the UCI repository are easy to learn with very simple representations
\cite{holte/93}; therefore those data sets may not be ideal for our
purposes.  Thus we created a number of artificial data sets that
present different problems for learning, and for which we know the
``correct'' concept definition.  This allows us to quantify more
precisely how the parameters of our algorithm affect its performance.

A second purpose of this experiment is to compare OC1's search
strategy with that of two existing oblique decision tree induction
systems -- LMDT \cite{brodley/utgoff/92} and SADT
\cite{heath/etal/93}.  We show that the quality of trees induced by
OC1 is as good as, if not better than, that of the trees induced by
these existing systems on three artificial domains. We also show that
OC1 achieves a good balance between amount of effort expended in
search and the quality of the tree induced.

Both LMDT and SADT used information gain for this experiment. However,
we did not change OC1's default measure (the Twoing Rule) because we
observed, in experiments not reported here, that OC1 with information
gain does not produce significantly different results.  The maximum
number of successive, unproductive perturbations allowed at any node
was set at 10000 for SADT.  For all other parameters, we used default
settings provided with the systems.

\subsubsection{Description of Artificial Data}
\label{section:artificialdata}

\paragraph{LS10}
The LS10 data set has 2000 instances divided into two categories.  Each
instance is described by ten attributes $x_1$,\dots,$x_{10}$, whose
values are uniformly distributed in the range [0,1].  The data is
linearly separable with a 10-D hyperplane (thus the name LS10) defined
by the equation \(x_1+x_2+x_3+x_4+x_5 < x_6+x_7+x_8+x_9+x_{10}\).  The
instances were all generated randomly and labelled according to which
side of this hyperplane they fell on.  Because oblique DT induction
methods intuitively should prefer a linear separator if one exists, it
is interesting to compare the various search techniques on this
data set where we know a separator exists.  The task is relatively
simple for lower dimensions, so we chose 10-dimensional data to make
it more difficult.

\paragraph{POL}
This data set is shown in Figure~\ref{figure:artificialdata}. It has
2000 instances in two dimensions, again divided into two categories.
The underlying concept is a set of four parallel oblique lines (thus
the name POL), dividing the instances into five homogeneous regions.  
This concept is more difficult to learn than a single
linear separator, but the minimal-size tree is still quite small.

\paragraph{RCB}
RCB stands for ``rotated checker board''; this data set has been the
subject of other experiments on hard classification problems for
decision trees \cite{murthy/salzberg/94}.  The data set, shown in
Figure~\ref{figure:artificialdata}, has 2000 instances in 2-D, each
belonging to one of eight categories.  This concept is difficult to
learn for any axis-parallel method, for obvious reasons.  It is also
quite difficult for oblique methods, for several reasons.  The
biggest problem is that the ``correct'' root node, as shown in the
figure, does not separate out any class by itself.  Some impurity
measures (such as Sum Minority) will fail miserably on this problem,
although others (e.g., the Twoing Rule) work much better.  Another
problem is that a deterministic coefficient perturbation algorithm 
can get stuck in local minima in many places on this data set.
\begin{figure}
\vspace{2.0in}
\special{psfile="pol.ps" hscale=25 vscale=25 hoffset=70 voffset=-10}
\special{psfile="rcb.ps" hscale=25 vscale=25 hoffset=200 voffset=-10}
\caption{The POL and RCB data sets}
\label{figure:artificialdata}
\vspace*{-.2in}
\end{figure}

Table \ref{table:2} summarizes the results of this experiment in
three smaller tables, one for each data set.
\begin{table}
\begin{center}
\begin{tabular}{|r|ccc|} \hline \hline
\multicolumn{4}{|c|}{Linearly Separable 10-D (LS10) data} \\ \hline 
R:J & Accuracy & Size  & Hyperplanes \\ \hline
0:0   & 89.8$\pm$1.2 & 67.0$\pm$5.8 & 2756 \\
0:20  & 91.5$\pm$1.5 & 55.2$\pm$7.0 & 3824 \\
20:0  & 95.0$\pm$0.6 & 25.6$\pm$2.4 & 24913 \\
20:20 & 97.2$\pm$0.7 & 13.9$\pm$3.2 & 30366 \\ \hline
LMDT  & 99.7$\pm$0.2 &  2.2$\pm$0.5 & 9089 \\ \hline
SADT  & 95.2$\pm$1.8 & 15.5$\pm$5.7 & 349067 \\ \hline  \hline
\multicolumn{4}{|c|}{Parallel Oblique Lines (POL) data} \\ \hline 
R:J & Accuracy & Size  & Hyperplanes \\ \hline
0:0   & 98.3$\pm$0.3 & 21.6$\pm$1.9 & 164 \\
0:20  & 99.3$\pm$0.2 &  9.0$\pm$1.0 & 360 \\
20:0  & 99.1$\pm$0.2 & 14.2$\pm$1.1 & 3230 \\
20:20 & 99.6$\pm$0.1 &  5.5$\pm$0.3 & 4852 \\ \hline
LMDT  & 89.6$\pm$10.2& 41.9$\pm$19.2& 1732 \\ \hline
SADT  & 99.3$\pm$0.4 &  8.4$\pm$2.1 & 85594 \\ \hline  \hline
\multicolumn{4}{|c|}{Rotated Checker Board (RCB) data} \\ \hline 
R:J & Accuracy & Size  & Hyperplanes \\ \hline
0:0   & 98.4$\pm$0.2 & 35.5$\pm$1.4 & 573 \\
0:20  & 99.3$\pm$0.3 & 19.7$\pm$0.8 & 1778 \\
20:0  & 99.6$\pm$0.2 & 12.0$\pm$1.4 & 6436 \\
20:20 & 99.8$\pm$0.1 &  8.7$\pm$0.4 & 11634 \\ \hline 
LMDT  & 95.7$\pm$2.3 & 70.1$\pm$9.6 & 2451 \\ \hline
SADT  & 97.9$\pm$1.1 & 32.5$\pm$4.9 & 359112 \\ \hline \hline
\end{tabular}
\caption{The effect of randomization in OC1\@.  The first column, 
labelled R:J, shows the number of restarts (R) followed by the maximum
number of random jumps (J) attempted by OC1 at each local minimum.
Results with LMDT and SADT are included for comparison after the four
variants of OC1.  Size is average tree size measured by the number of
leaf nodes.  The third column shows the average number of
hyperplanes each algorithm considered while building one tree.}
\label{table:2}
\vspace*{-.4in}
\end{center}
\end{table}
In each smaller table, we compare four variants of OC1 with LMDT and
SADT\@.  The different results for OC1 were obtained by varying both
the number of restarts and the number of random jumps.  When random
jumps were used, up to twenty random jumps were tried at each local
minimum.  As soon as one was found that improved the impurity of the
current hyperplane, the algorithm moved the hyperplane and started
running the deterministic perturbation procedure again.  If none of
the 20 random jumps improved the impurity, the search halted and
further restarts (if any) were tried.  The same training and test
partitions were used for all methods for each cross-validation run
(recall that the results are an average of ten 5-fold CVs).  The
trees were not pruned for any of the algorithms, because the data were
noise-free and furthermore the emphasis was on search.

Table \ref{table:2} also includes the number of hyperplanes considered
by each algorithm while building a complete tree.  Note that for OC1
and SADT, the number of hyperplanes considered is generally much
larger than the number of perturbations actually made, because both
these algorithms compare newly generated hyperplanes to existing
hyperplanes before adjusting an existing one.  Nevertheless, this
number is a good estimate of much effort each algorithm expends,
because every new hyperplane must be evaluated according to the
impurity measure.  For LMDT, the number of hyperplanes considered is
identical to the actual number of perturbations.

\subsubsection{Discussion}

The OC1 results here are quite clear.  The first line of each table,
labelled 0:0, gives the accuracies and tree sizes when no
randomization is used --- this variant is very similar to the CART-LC
algorithm.  As we increase the use of randomization, accuracy
increases while tree size decreases, which is exactly the result we
had hoped for when we decided to introduce randomization into the
method.

Looking more closely at the tables, we can ask about the effect of
random jumps alone.  This is illustrated in the second line (0:20) of
each table, which attempted up to 20 random jumps at each local
minimum and no restarts.  Accuracy increased by 1-2\% on each domain,
and tree size decreased dramatically, roughly by a factor of two, in
the POL and RCB domains.  Note that because there is no noise in these
domains, very high accuracies should be expected.  Thus increases of
more than a few percent in accuracy are not possible.

Looking at the third line of each sub-table in Table~\ref{table:2}, we
see the effect of multiple restarts on OC1.  With 20 restarts but no
random jumps to escape local minima, the improvement is even more
noticeable for the LS10 data than when random jumps alone were used.
For this data set, accuracy jumped significantly, from 89.8 to 95.0\%,
while tree size dropped from 67 to 26 nodes.  For the POL and RCB
data, the improvements were comparable to those obtained with random
jumps.  For the RCB data, tree size dropped by a factor of 3 (from 36
leaf nodes to 12 leaf nodes) while accuracy increased from 98.4 to
99.6\%.

The fourth line of each table shows the effect of both the randomized
steps.  Among the OC1 entries, this line has both the highest
accuracies and the smallest trees for all three data sets, so it is
clear that randomization is a big win for these kinds of problems.  In
addition, note that the smallest tree for the RCB data should have
eight leaf nodes, and OC1's average trees, without pruning, had just
8.7 leaf nodes.  It is clear that for this data set, which we thought
was the most difficult one, OC1 came very close to finding the optimal
tree on nearly every run. (Recall that numbers in the table are the
average of 10 5-fold CV experiments; i.e., an average of 50 decision
trees.)  The LS10 data show how difficult it can be to find a very
simple concept in higher dimensions---the optimal tree there is just a
single hyperplane (two nodes), but OC1 was unable to find it with the
current parameter settings.\footnote {In a separate experiment, we
found that OC1 consistently finds the linear separator for the LS10
data when 10 restarts and 200 random jumps are used.} The POL data
required a minimum of 5 leaf nodes, and OC1 found this minimal-size
tree most of the time, as can be seen from the table.  Although not
shown in the Table, OC1 using Sum Minority performed better for the
POL data than the Twoing Rule or any other impurity measure; i.e., it
found the correct tree using less time.

The results of LMDT and SADT on this data lead to some interesting
insights.  Not surprisingly, LMDT does very well on the linearly
separable (LS10) data, and does not require an inordinate amount of
search.  Clearly, if the data is linearly separable, one should use
a method such as LMDT or linear programming.  OC1 and SADT have
difficulty finding the linear separator, although in our experiments
OC1 did eventually find it, given sufficient time.

On the other hand, for both of the non-linearly separable data sets,
LMDT produces much larger trees that are significantly less accurate
than those produced by OC1 and SADT.  Even the deterministic variant
of OC1 (using zero restarts and zero random jumps) outperforms LMDT on
these problems, with much less search.  

Although SADT sometimes produces very accurate trees, its main
weakness was the enormous amount of search time it required, roughly
10-20 times greater than OC1 even using the 20:20 setting.  One
explanation of OC1's advantage is its use of directed search, as
opposed to the strictly random search used by simulated annealing.
Overall, Table \ref{table:2} shows that OC1's use of randomization was
quite effective for the non-linearly separable data.

It is natural to ask {\it why} randomization helps OC1 in the task of
inducing decision trees.  Researchers in combinatorial optimization
have observed that randomized search usually succeeds when the search
space holds an abundance of good solutions \cite{gupta/etal/94}.
Furthermore, randomization can improve upon deterministic search when
many of the local maxima in a search space lead to poor solutions.  In
OC1's search space, a local maximum is a hyperplane that cannot be
improved by the deterministic search procedure, and a ``solution'' is
a complete decision tree.  If a significant fraction of local maxima
lead to bad trees, then algorithms that stop at the first local
maximum they encounter will perform poorly.  Because randomization
allows OC1 to consider many different local maxima, if a modest
percentage of these maxima lead to good trees, then it has a good
chance of finding one of those trees.  Our experiments with OC1 thus
far indicate that the space of oblique hyperplanes usually contains
numerous local maxima, and that a substantial percentage of these
locally good hyperplanes lead to good decision trees.

\section{Conclusions and Future Work}
\label{section:conclude}

This paper has described OC1, a new system for constructing oblique
decision trees.  We have shown experimentally that OC1 can produce
good classifiers for a range of real-world and artificial domains.  We
have also shown how the use of randomization improves upon the
original algorithm proposed by Breiman et al.~(1984), without
significantly increasing the computational cost of the algorithm.

The use of randomization might also be beneficial for axis-parallel
tree methods.  Note that although they do find the optimal test (with
respect to an impurity measure) for each node of a tree, the
complete tree may not be optimal: as is well known, the problem of
finding the smallest tree is NP-Complete \cite{hyafil/rivest/76}.
Thus even axis-parallel decision tree methods do not produce ``ideal''
decision trees.  Quinlan has suggested that his windowing algorithm
might be used as a way of introducing randomization into C4.5, even
though the algorithm was designed for another purpose
\cite{quinlan/93}.  (The windowing algorithm selects a 
random subset of the training data and builds a tree using that.)  We
believe that randomization is a powerful tool in the context of
decision trees, and our experiments are just one example of how it
might be exploited.  We are in the process of conducting further
experiments to quantify more accurately the effects of different forms
of randomization.

It should be clear that the ability to produce oblique splits at a
node broadens the capabilities of decision tree algorithms, especially
as regards domains with numeric attributes.  Of course, axis-parallel
splits are simpler, in the sense that the description of the split
only uses one attribute at each node.  OC1 uses oblique splits only
when their impurity is less than the impurity of the best
axis-parallel split; however, one could easily penalize the additional
complexity of an oblique split further.  This remains an open area for
further research.  A more general point is that if the domain is best
captured by a tree that uses oblique hyperplanes, it is desirable to
have a system that can generate that tree.  We have shown that for
some problems, including those used in our experiments, OC1 builds
small decision trees that capture the domain well.

\appendix
\section*{Appendix A. Complexity Analysis of OC1}
\label{appendix:1}

In the following, we show that OC1 runs efficiently even in the worst
case.  For a data set with $n$ examples (points) and $d$ attributes
per example, OC1 uses at most $O(dn^2 \log n)$ time.  We assume $n >
d$ for our analysis.

For the analysis here, we assume the coefficients of a hyperplane are
adjusted in sequential order (the Seq method described in the paper).
The number of restarts at a node will be $r$, and the number of random
jumps tried will be $j$.  Both $r$ and $j$ are constants, fixed in
advance of running the algorithm.

Initializing the hyperplane to a random position takes just $O(d)$
time.  We need to consider first the maximum amount of work OC1 can do
before it finds a new location for the hyperplane.  Then we need to
consider how many times it can move the hyperplane.
\begin{enumerate}
\item Attempting to perturb the first coefficient ($a_1$) takes 
$O(dn + n \log n)$ time.  Computing $U_i$'s for all the points
(equation~\ref{equation:uj}) requires $O(dn)$ time, and sorting the
$U_i$'s takes $O(n \log n)$.  This gives us $O(dn + n \log n)$ work.

\item If perturbing $a_1$ does not improve things, we try to perturb $a_2$.
Computing all the new $U_i$'s will take just $O(n)$ time because only
one term is different for each $U_i$.  Re-sorting will take $O(n \log n)$,
so this step takes $O(n) + O(n \log n) = O(n \log n)$ time.  

\item Likewise $a_3, \ldots, a_d$ will each take $O(n \log n)$
additional time, assuming we still have not found a better hyperplane
after checking each coefficient.  Thus the total time to cycle through
and attempt to perturb all these additional coefficients is 
$(d-1) * O(n \log n) = O(dn \log n)$.

\item Summing up, the time to cycle through all coefficients is
$O(dn \log n) + O(dn + n \log n) = O(dn \log n)$.

\item If none of the coefficients improved the split, then we attempt to
make up to $j$ random jumps.  Since $j$ is a constant, we will just
consider $j=1$ for our analysis.  This step involves choosing a
random vector and running the perturbation algorithm to solve
for $\alpha$, as explained in Section~\ref{section:rand}.
As before, we need to compute a set of $U_i$'s and sort them,
which takes $O(dn + n \log n)$ time.  Because this amount of time
is dominated by the time to adjust all the coefficients, 
the total time so far is still $O(dn \log n)$.  This is the most
time OC1 can spend at a node before either halting or finding
an improved hyperplane.

\item Assuming OC1 is using the Sum Minority or Max Minority error 
measure, it can only reduce the impurity of the hyperplane $n$ times.
This is clear because each improvement means one more example will be
correctly classified by the new hyperplane.  Thus the total amount of
work at a node is limited to $n * O(dn \log n) = O(dn^2 \log n)$.
(This analysis extends, with at most linear cost factors, to
Information Gain, Gini Index and Twoing Rule when there are two
categories.  It will not apply to a measure that, for example, uses
the distances of mis-classified objects to the hyperplane.)  In
practice, we have found that the number of improvements per node is
much smaller than $n$.
\end{enumerate}

Assuming that OC1 only adjusts a hyperplane when it improves the
impurity measure, it will do $O(dn^2 \log n)$ work in the worst case.

However, OC1 allows a certain number of adjustments to the hyperplane
that do not improve the impurity, although it will never accept a
change that worsens the impurity.  The number allowed is
determined by a constant known as ``stagnant-perturbations''.  Let
this value be $s$.  This works as follows.

Each time OC1 finds a new hyperplane that improves on the old one,
it resets a counter to zero.  It will move the new hyperplane to
a different location that has {\it equal} impurity at most $s$
times.  After each of these moves it repeats the perturbation 
algorithm.  Whenever impurity is reduced, it re-starts the
counter and again allows $s$ moves to equally good locations.
Thus it is clear that this feature just increases the worst-case
complexity of OC1 by a constant factor, $s$.

Finally, note that the overall cost of OC1 is also $O(dn^2 \log n)$,
i.e., this is an upper bound on the total running time of OC1
independent of the size of the tree it ends up creating.  (This upper
bound applies to Sum Minority and Max Minority; an open question is
whether a similar upper bound can be proven for Information Gain or
the Gini Index.)  Thus the worst-case asymptotic complexity of our
system is comparable to that of systems that construct axis-parallel
decision trees, which have $O(dn^2)$ worst-case complexity.  To sketch
the intuition that leads to this bound, let $G$ be the total impurity
summed over all leaves in a partially constructed tree (i.e., the sum
of currently misclassified points in the tree).  Now observe that each
time we run the perturbation algorithm on any node in the tree, we
either halt or improve $G$ by at least one unit.  The worst-case
analysis for one node is realized when the perturbation algorithm is run
once for every one of the $n$ examples, but when this happens, there
would no longer be any mis-classified examples and the tree would be
complete.

\section*{Appendix B. Definitions of impurity measures available in OC1}
\label{appendix:2}
In addition to the Twoing Rule defined in the text, OC1 contains
built-in definitions of five additional impurity measures, defined as
follows.  In each of the following definitions, the set of examples
$T$ at the node about to be split contains $n$ ($>0$) instances that
belong to one of $k$ categories.  (Initially this set is the entire
training set.)  A hyperplane $H$ divides $T$ into two non-overlapping
subsets $T_L$ and $T_R$ (i.e., left and right).  $L_j$ and $R_j$ are
the number of instances of category $j$ in $T_L$ and $T_R$
respectively.  All the impurity measures initially check to see if
$T_L$ and $T_R$ are homogeneous (i.e., all examples belong to the same
category), and if so return minimum (zero) impurity.

\paragraph{Information Gain.}
This measure of information gained from a particular split was
popularized in the context of decision trees by Quinlan (1986).
Quinlan's definition makes information gain a goodness measure; i.e.,
something to maximize.  Because OC1 attempts to minimize whatever
impurity measure it uses, we use the reciprocal of the standard value
of information gain in the OC1 implementation.

\paragraph{Gini Index.}
The Gini Criterion (or Index) was proposed for decision trees by
Breiman et al.\ (1984).  The Gini Index as originally
defined measures the probability of misclassification of a set of
instances, rather than the impurity of a split. We implement the
following variation:
\[ \mbox{GiniL} = 1.0 - \sum_{i=1}^{k} {(L_i / |T_L|)^2} \]  
\[ \mbox{GiniR} = 1.0 - \sum_{i=1}^{k} {(R_i / |T_R|)^2} \]  
\[ \mbox{Impurity} = (|T_L| * \mbox{GiniL} + |T_R| * \mbox{GiniR})/ n  \]
where GiniL is the Gini Index on the ``left'' side of the hyperplane and 
GiniR is that on the right.

\paragraph{Max Minority.}
The measures Max Minority, Sum Minority and Sum Of Variances were
defined in the context of decision trees by Heath, Kasif, and
Salzberg~\citeyear{heath/etal/93}.\footnote{Sum Of Variances was
called Sum of Impurities by Heath et al.}  Max Minority has the
theoretical advantage that a tree built minimizing this measure will
have depth at most $\log n$. Our experiments indicated that this is
not a great advantage in practice: seldom do other impurity measures
produce trees substantially deeper than those produced with Max
Minority.  The definition is:
\[ \mbox{MinorityL} = \sum_{i=1, i \neq \max L_i}^{k} {L_i}  \]
\[ \mbox{MinorityR} = \sum_{i=1, i \neq \max R_i}^{k} {R_i}  \]
\[ \mbox{Max Minority} =  \max (\mbox{MinorityL}, \mbox{MinorityR}) \]

\paragraph{Sum Minority.}
This measure is very similar to Max Minority.  If MinorityL and
MinorityR are defined as for the Max Minority measure, then Sum
Minority is just the sum of these two values.  This measure is the
simplest way of quantifying impurity, as it simply counts the number
of misclassified instances.

Though Sum Minority performs well on some domains, it has some obvious
flaws.  As one example, consider a domain in which $n=100,d=1$, and
$k=2$ (i.e., 100 examples, 1 numeric attribute, 2 classes).  Suppose
that when the examples are sorted according to the single attribute,
the first 50 instances belong to category 1, followed by 24 instances
of category 2, followed by 26 instances of category 1.  Then {\em all}
possible splits for this distribution have a sum minority of 24.
Therefore it is impossible when using Sum Minority to distinguish
which split is preferable, although splitting at the alternations
between categories is clearly better.

\paragraph{Sum Of Variances.}
The definition of this measure is:
\[ \mbox{VarianceL} = \sum_{i=1}^{|T_L|} {(Cat(T_{L_i}) - 
		   \sum_{j=1}^{|T_L|} Cat(T_{L_j}) / |T_L|)^2}\]
\[ \mbox{VarianceR} = \sum_{i=1}^{|T_R|} {(Cat(T_{R_i}) - 
		   \sum_{j=1}^{|T_R|} Cat(T_{R_j}) / |T_R|)^2}\]
\[ \mbox{Sum of Variances} = \mbox{VarianceL} + \mbox{VarianceR} \]
where Cat($T_i$) is the category of instance $T_i$.  As this measure
is computed using the actual class labels, it is easy to see that the
impurity computed varies depending on how numbers are assigned to the
classes.  For instance, if $T_1$ consists of 10 points of category 1
and 3 points of category 2, and if $T_2$ consists of 10 points of
category 1 and 3 points of category 5, then the Sum Of Variances
values are different for $T_1$ and $T_2$.  To avoid this problem, OC1
uniformly reassigns category numbers according to the frequency of
occurrence of each category at a node before computing the Sum Of
Variances.

\section*{Acknowledgements}
The authors thank Richard Beigel of Yale University for suggesting the
idea of jumping in a random direction.  Thanks to Wray Buntine of Nasa
Ames Research Center for providing the IND 2.1 package, to Carla
Brodley for providing the LMDT code, and to David Heath for providing
the SADT code and for assisting us in using it.  Thanks also to three
anonymous reviewers for many helpful suggestions.  This material is
based upon work supported by the National Science foundation under
Grant Nos.\ IRI-9116843, IRI-9223591, and IRI-9220960.

\bibliographystyle{theapa}

\begin{thebibliography}{}

\bibitem[\protect\BCAY{Aha}{Aha}{1990}]{aha/90}
Aha, D. \BBOP1990\BBCP.
\newblock {\Bem A Study of Instance-Based Algorithms for Supervised Learning:
  Mathematical, empirical and psychological evaluations}.
\newblock Ph.D.\ thesis, Department of Information and Computer Science,
  University of California, Irvine.

\bibitem[\protect\BCAY{Almuallin \BBA\ Dietterich}{Almuallin \BBA\
  Dietterich}{1991}]{almuallin/dietterich/91}
Almuallin, H.\BBACOMMA\  \BBA\ Dietterich, T. \BBOP1991\BBCP.
\newblock \BBOQ Learning with many irrelevant features\BBCQ\
\newblock In {\Bem Proceedings of the Ninth National Conference on Artificial
  Intelligence}, \BPGS\ 547--552. San Jose, CA.

\bibitem[\protect\BCAY{Belsley}{Belsley}{1980}]{belsley/80}
Belsley, D. \BBOP1980\BBCP.
\newblock {\Bem Regression Diagnostics: Identifying Influential Data and
  Sources of Collinearity}.
\newblock Wiley \& Sons, New York.

\bibitem[\protect\BCAY{Bennett \BBA\ Mangasarian}{Bennett \BBA\
  Mangasarian}{1992}]{bennett/mangasarian/92}
Bennett, K.\BBACOMMA\  \BBA\ Mangasarian, O. \BBOP1992\BBCP.
\newblock \BBOQ Robust linear programming discrimination of two linearly
  inseparable sets\BBCQ\
\newblock {\Bem Optimization Methods and Software}, {\Bem 1}, 23--34.

\bibitem[\protect\BCAY{Bennett \BBA\ Mangasarian}{Bennett \BBA\
  Mangasarian}{1994a}]{bennett/mangasarian/94a}
Bennett, K.\BBACOMMA\  \BBA\ Mangasarian, O. \BBOP1994a\BBCP.
\newblock \BBOQ Multicategory discrimination via linear programming\BBCQ\
\newblock {\Bem Optimization Methods and Software}, {\Bem 3}, 29--39.

\bibitem[\protect\BCAY{Bennett \BBA\ Mangasarian}{Bennett \BBA\
  Mangasarian}{1994b}]{bennett/mangasarian/94b}
Bennett, K.\BBACOMMA\  \BBA\ Mangasarian, O. \BBOP1994b\BBCP.
\newblock \BBOQ Serial and parallel multicategory discrimination\BBCQ\
\newblock {\Bem SIAM Journal on Optimization}, {\Bem 4\/}(4).

\bibitem[\protect\BCAY{Blum \BBA\ Rivest}{Blum \BBA\
  Rivest}{1988}]{blum/rivest/88}
Blum, A.\BBACOMMA\  \BBA\ Rivest, R. \BBOP1988\BBCP.
\newblock \BBOQ Training a 3-node neural network is {NP}-complete\BBCQ\
\newblock In {\Bem Proceedings of the 1988 Workshop on Computational Learning
  Theory}, \BPGS\ 9--18. Boston, MA. Morgan Kaufmann.

\bibitem[\protect\BCAY{Breiman, Friedman, Olshen, \BBA\ Stone}{Breiman
  et~al.}{1984}]{breiman/etal/84}
Breiman, L., Friedman, J., Olshen, R., \BBA\ Stone, C. \BBOP1984\BBCP.
\newblock {\Bem Classification and Regression Trees}.
\newblock Wadsworth International Group.

\bibitem[\protect\BCAY{Brent}{Brent}{1991}]{brent/91}
Brent, R.~P. \BBOP1991\BBCP.
\newblock \BBOQ Fast training algorithms for multilayer neural nets\BBCQ\
\newblock {\Bem {IEEE} Transactions on Neural Networks}, {\Bem 2\/}(3),
  346--354.

\bibitem[\protect\BCAY{Brodley \BBA\ Utgoff}{Brodley \BBA\
  Utgoff}{1992}]{brodley/utgoff/92}
Brodley, C.~E.\BBACOMMA\  \BBA\ Utgoff, P.~E. \BBOP1992\BBCP.
\newblock \BBOQ Multivariate versus univariate decision trees\BBCQ\
\newblock \BTR\ COINS CR 92-8, Dept. of Computer Science, University of
  Massachusetts at Amherst.

\bibitem[\protect\BCAY{Brodley \BBA\ Utgoff}{Brodley \BBA\
  Utgoff}{1994}]{brodley/utgoff/94}
Brodley, C.~E.\BBACOMMA\  \BBA\ Utgoff, P.~E. \BBOP1994\BBCP.
\newblock \BBOQ Multivariate decision trees\BBCQ\
\newblock {\Bem Machine Learning}, {\Bem to appear}.

\bibitem[\protect\BCAY{Buntine}{Buntine}{1992}]{buntine/92}
Buntine, W. \BBOP1992\BBCP.
\newblock \BBOQ Tree classification software\BBCQ\
\newblock Technology 2002: The Third National Technology Transfer Conference
  and Exposition.

\bibitem[\protect\BCAY{Buntine \BBA\ Niblett}{Buntine \BBA\
  Niblett}{1992}]{buntine/niblett/92}
Buntine, W.\BBACOMMA\  \BBA\ Niblett, T. \BBOP1992\BBCP.
\newblock \BBOQ A further comparison of splitting rules for decision-tree
  induction\BBCQ\
\newblock {\Bem Machine Learning}, {\Bem 8}, 75--85.

\bibitem[\protect\BCAY{Cardie}{Cardie}{1993}]{cardie/93}
Cardie, C. \BBOP1993\BBCP.
\newblock \BBOQ Using decision trees to improve case-based learning\BBCQ\
\newblock In {\Bem Proceedings of the Tenth International Conference on Machine
  Learning}, \BPGS\ 25--32. University of Massachusetts, Amherst.

\bibitem[\protect\BCAY{Cestnik, Kononenko, \BBA\ Bratko}{Cestnik
  et~al.}{1987}]{cestnik/etal/87}
Cestnik, G., Kononenko, I., \BBA\ Bratko, I. \BBOP1987\BBCP.
\newblock \BBOQ Assistant 86: A knowledge acquisition tool for sophisticated
  users\BBCQ\
\newblock In Bratko, I.\BBACOMMA\  \BBA\ Lavrac, N.\BEDS, {\Bem Progress in
  Machine Learning}. Sigma Press.

\bibitem[\protect\BCAY{Cios \BBA\ Liu}{Cios \BBA\ Liu}{1992}]{cios/liu/92}
Cios, K.~J.\BBACOMMA\  \BBA\ Liu, N. \BBOP1992\BBCP.
\newblock \BBOQ A machine learning method for generation of a neural network
  architecture: A continuous {ID}3 algorithm\BBCQ\
\newblock {\Bem {IEEE} Transactions on Neural Networks}, {\Bem 3\/}(2),
  280--291.

\bibitem[\protect\BCAY{Cohen}{Cohen}{1993}]{cohen/93}
Cohen, W. \BBOP1993\BBCP.
\newblock \BBOQ Efficient pruning methods for separate-and-conquer rule
  learning systems\BBCQ\
\newblock In {\Bem Proceedings of the 13th International Joint Conference on
  Artificial Intelligence}, \BPGS\ 988--994. Morgan Kaufmann.

\bibitem[\protect\BCAY{Fayyad \BBA\ Irani}{Fayyad \BBA\
  Irani}{1992}]{fayyad/irani/92b}
Fayyad, U.~M.\BBACOMMA\  \BBA\ Irani, K.~B. \BBOP1992\BBCP.
\newblock \BBOQ The attribute specification problem in decision tree
  generation\BBCQ\
\newblock In {\Bem Proceedings of the Tenth National Conference on Artificial
  Intelligence}, \BPGS\ 104--110. San Jose CA. AAAI Press.

\bibitem[\protect\BCAY{Frean}{Frean}{1990}]{frean/90}
Frean, M. \BBOP1990\BBCP.
\newblock {\Bem Small Nets and Short Paths: Optimising neural computation}.
\newblock Ph.D.\ thesis, Centre for Cognitive Science, University of Edinburgh.

\bibitem[\protect\BCAY{Gupta, Smolka, \BBA\ Bhaskar}{Gupta
  et~al.}{1994}]{gupta/etal/94}
Gupta, R., Smolka, S., \BBA\ Bhaskar, S. \BBOP1994\BBCP.
\newblock \BBOQ On randomization in sequential and distributed algorithms\BBCQ\
\newblock {\Bem ACM Computing Surveys}, {\Bem 26\/}(1), 7--86.

\bibitem[\protect\BCAY{Hampson \BBA\ Volper}{Hampson \BBA\
  Volper}{1986}]{hampson/volper/86}
Hampson, S.\BBACOMMA\  \BBA\ Volper, D. \BBOP1986\BBCP.
\newblock \BBOQ Linear function neurons: Structure and training\BBCQ\
\newblock {\Bem Biological Cybernetics}, {\Bem 53}, 203--217.

\bibitem[\protect\BCAY{Harrison \BBA\ Rubinfeld}{Harrison \BBA\
  Rubinfeld}{1978}]{harrison/rubinfeld/78}
Harrison, D.\BBACOMMA\  \BBA\ Rubinfeld, D. \BBOP1978\BBCP.
\newblock \BBOQ Hedonic prices and the demand for clean air\BBCQ\
\newblock {\Bem Journal of Environmental Economics and Management}, {\Bem 5},
  81--102.

\bibitem[\protect\BCAY{Hassibi \BBA\ Stork}{Hassibi \BBA\
  Stork}{1993}]{hassibi/stork/93}
Hassibi, B.\BBACOMMA\  \BBA\ Stork, D. \BBOP1993\BBCP.
\newblock \BBOQ Second order derivatives for network pruning: optimal brain
  surgeon\BBCQ\
\newblock In {\Bem Advances in Neural Information Processing Systems 5}, \BPGS\
  164--171. Morgan Kaufmann, San Mateo, CA.

\bibitem[\protect\BCAY{Heath}{Heath}{1992}]{heath/92}
Heath, D. \BBOP1992\BBCP.
\newblock {\Bem A Geometric Framework for Machine Learning}.
\newblock Ph.D.\ thesis, Johns Hopkins University, Baltimore, Maryland.

\bibitem[\protect\BCAY{Heath, Kasif, \BBA\ Salzberg}{Heath
  et~al.}{1993a}]{heath/etal/93b}
Heath, D., Kasif, S., \BBA\ Salzberg, S. \BBOP1993a\BBCP.
\newblock \BBOQ k-{DT}: A multi-tree learning method\BBCQ\
\newblock In {\Bem Proceedings of the Second International Workshop on
  Multistrategy Learning}, \BPGS\ 138--149. Harpers Ferry, WV. George Mason
  University.

\bibitem[\protect\BCAY{Heath, Kasif, \BBA\ Salzberg}{Heath
  et~al.}{1993b}]{heath/etal/93}
Heath, D., Kasif, S., \BBA\ Salzberg, S. \BBOP1993b\BBCP.
\newblock \BBOQ Learning oblique decision trees\BBCQ\
\newblock In {\Bem Proceedings of the 13th International Joint Conference on
  Artificial Intelligence}, \BPGS\ 1002--1007. Chambery, France. Morgan
  Kaufmann.

\bibitem[\protect\BCAY{Herman \BBA\ Yeung}{Herman \BBA\
  Yeung}{1992}]{herman/yeung/92}
Herman, G.~T.\BBACOMMA\  \BBA\ Yeung, K.~D. \BBOP1992\BBCP.
\newblock \BBOQ On piecewise-linear classification\BBCQ\
\newblock {\Bem {IEEE} Transactions on Pattern Analysis and Machine
  Intelligence}, {\Bem 14\/}(7), 782--786.

\bibitem[\protect\BCAY{Holte}{Holte}{1993}]{holte/93}
Holte, R. \BBOP1993\BBCP.
\newblock \BBOQ Very simple classification rules perform well on most commonly
  used datasets\BBCQ\
\newblock {\Bem Machine Learning}, {\Bem 11\/}(1), 63--90.

\bibitem[\protect\BCAY{Hyafil \BBA\ Rivest}{Hyafil \BBA\
  Rivest}{1976}]{hyafil/rivest/76}
Hyafil, L.\BBACOMMA\  \BBA\ Rivest, R.~L. \BBOP1976\BBCP.
\newblock \BBOQ Constructing optimal binary decision trees is
  {NP}-complete\BBCQ\
\newblock {\Bem Information Processing Letters}, {\Bem 5\/}(1), 15--17.

\bibitem[\protect\BCAY{Kira \BBA\ Rendell}{Kira \BBA\
  Rendell}{1992}]{kira/rendell/92}
Kira, K.\BBACOMMA\  \BBA\ Rendell, L. \BBOP1992\BBCP.
\newblock \BBOQ A practical approach to feature selection\BBCQ\
\newblock In {\Bem Proceedings of the Ninth International Conference on Machine
  Learning}, \BPGS\ 249--256. Aberdeen, Scotland. Morgan Kaufmann.

\bibitem[\protect\BCAY{Kirkpatrick, Gelatt, \BBA\ Vecci}{Kirkpatrick
  et~al.}{1983}]{kirkpatrick/etal/83}
Kirkpatrick, S., Gelatt, C., \BBA\ Vecci, M. \BBOP1983\BBCP.
\newblock \BBOQ Optimization by simulated annealing\BBCQ\
\newblock {\Bem Science}, {\Bem 220\/}(4598), 671--680.

\bibitem[\protect\BCAY{Kodratoff \BBA\ Manago}{Kodratoff \BBA\
  Manago}{1987}]{kodratoff/manago/87}
Kodratoff, Y.\BBACOMMA\  \BBA\ Manago, M. \BBOP1987\BBCP.
\newblock \BBOQ Generalization and noise\BBCQ\
\newblock {\Bem International Journal of Man-Machine Studies}, {\Bem 27},
  181--204.

\bibitem[\protect\BCAY{Langley \BBA\ Sage}{Langley \BBA\
  Sage}{1993}]{langley/sage/93}
Langley, P.\BBACOMMA\  \BBA\ Sage, S. \BBOP1993\BBCP.
\newblock \BBOQ Scaling to domains with many irrelevant features\BBCQ\
\newblock Learning Systems Department, Siemens Corporate Research, Princeton,
  NJ.

\bibitem[\protect\BCAY{Mangasarian, Setiono, \BBA\ Wolberg}{Mangasarian
  et~al.}{1990}]{mangasarian/etal/90}
Mangasarian, O., Setiono, R., \BBA\ Wolberg, W. \BBOP1990\BBCP.
\newblock \BBOQ Pattern recognition via linear programming: Theory and
  application to medical diagnosis\BBCQ\
\newblock In {\Bem SIAM Workshop on Optimization}.

\bibitem[\protect\BCAY{Mingers}{Mingers}{1989a}]{mingers/89b}
Mingers, J. \BBOP1989a\BBCP.
\newblock \BBOQ An empirical comparison of pruning methods for decision tree
  induction\BBCQ\
\newblock {\Bem Machine Learning}, {\Bem 4\/}(2), 227--243.

\bibitem[\protect\BCAY{Mingers}{Mingers}{1989b}]{mingers/89a}
Mingers, J. \BBOP1989b\BBCP.
\newblock \BBOQ An empirical comparison of selection measures for decision tree
  induction\BBCQ\
\newblock {\Bem Machine Learning}, {\Bem 3}, 319--342.

\bibitem[\protect\BCAY{Moret}{Moret}{1982}]{moret/82}
Moret, B.~M. \BBOP1982\BBCP.
\newblock \BBOQ Decision trees and diagrams\BBCQ\
\newblock {\Bem Computing Surveys}, {\Bem 14\/}(4), 593--623.

\bibitem[\protect\BCAY{Murphy \BBA\ Aha}{Murphy \BBA\ Aha}{1994}]{mlrepository}
Murphy, P.\BBACOMMA\  \BBA\ Aha, D. \BBOP1994\BBCP.
\newblock \BBOQ {UCI} repository of machine learning databases -- a
  machine-readable data repository\BBCQ\
\newblock Maintained at the Department of Information and Computer Science,
  University of California, Irvine. Anonymous {FTP} from ics.uci.edu in the
  directory pub/machine-learning-databases.

\bibitem[\protect\BCAY{Murthy, Kasif, Salzberg, \BBA\ Beigel}{Murthy
  et~al.}{1993}]{murthy/etal/93}
Murthy, S.~K., Kasif, S., Salzberg, S., \BBA\ Beigel, R. \BBOP1993\BBCP.
\newblock \BBOQ {OC1}: Randomized induction of oblique decision trees\BBCQ\
\newblock In {\Bem Proceedings of the Eleventh National Conference on
  Artificial Intelligence}, \BPGS\ 322--327. Washington, D.C. MIT Press.

\bibitem[\protect\BCAY{Murthy \BBA\ Salzberg}{Murthy \BBA\
  Salzberg}{1994}]{murthy/salzberg/94}
Murthy, S.~K.\BBACOMMA\  \BBA\ Salzberg, S. \BBOP1994\BBCP.
\newblock \BBOQ Using structure to improve decision trees\BBCQ\
\newblock \BTR\ JHU-94/12, Department of Computer Science, Johns Hopkins University.

\bibitem[\protect\BCAY{Niblett}{Niblett}{1986}]{niblett/86}
Niblett, T. \BBOP1986\BBCP.
\newblock \BBOQ Constructing decision trees in noisy domains\BBCQ\
\newblock In Bratko, I.\BBACOMMA\  \BBA\ Lavrac, N.\BEDS, {\Bem Progress in
  Machine Learning}. Sigma Press, England.

\bibitem[\protect\BCAY{Nilsson}{Nilsson}{1990}]{nilsson/90}
Nilsson, N. \BBOP1990\BBCP.
\newblock {\Bem Learning Machines}.
\newblock Morgan Kaufmann, San Mateo, CA.

\bibitem[\protect\BCAY{Odewahn, Stockwell, Pennington, Hum\-phreys, \BBA\
  Zumach}{Odewahn et~al.}{1992}]{odewahn/etal/92}
Odewahn, S., Stockwell, E., Pennington, R., Hum\-phreys, R., \BBA\ Zumach, W.
  \BBOP1992\BBCP.
\newblock \BBOQ Automated star-galaxy descrimination with neural networks\BBCQ\
\newblock {\Bem Astronomical Journal}, {\Bem 103\/}(1), 318--331.

\bibitem[\protect\BCAY{Pagallo}{Pagallo}{1990}]{pagallo/90}
Pagallo, G. \BBOP1990\BBCP.
\newblock {\Bem Adaptive Decision Tree Algorithms for Learning From Examples}.
\newblock Ph.D.\ thesis, University of California at Santa Cruz.

\bibitem[\protect\BCAY{Pagallo \BBA\ Haussler}{Pagallo \BBA\
  Haussler}{1990}]{pagallo/haussler/90}
Pagallo, G.\BBACOMMA\  \BBA\ Haussler, D. \BBOP1990\BBCP.
\newblock \BBOQ Boolean feature discovery in empirical learning\BBCQ\
\newblock {\Bem Machine Learning}, {\Bem 5\/}(1), 71--99.

\bibitem[\protect\BCAY{Quinlan}{Quinlan}{1983}]{quinlan/83}
Quinlan, J.~R. \BBOP1983\BBCP.
\newblock \BBOQ Learning efficient classification procedures and their
  application to chess end games\BBCQ\
\newblock In Michalski, R., Carbonell, J., \BBA\ Mitchell, T.\BEDS, {\Bem
  Machine Learning: An Artificial Intelligence Approach}. Morgan Kaufmann, San
  Mateo, CA.

\bibitem[\protect\BCAY{Quinlan}{Quinlan}{1986}]{quinlan/86}
Quinlan, J.~R. \BBOP1986\BBCP.
\newblock \BBOQ Induction of decision trees\BBCQ\
\newblock {\Bem Machine Learning}, {\Bem 1}, 81--106.

\bibitem[\protect\BCAY{Quinlan}{Quinlan}{1987}]{quinlan/87b}
Quinlan, J.~R. \BBOP1987\BBCP.
\newblock \BBOQ Simplifying decision trees\BBCQ\
\newblock {\Bem International Journal of Man-Machine Studies}, {\Bem 27},
  221--234.

\bibitem[\protect\BCAY{Quinlan}{Quinlan}{1993a}]{quinlan/93}
Quinlan, J.~R. \BBOP1993a\BBCP.
\newblock {\Bem C4.5: Programs for Machine Learning}.
\newblock Morgan Kaufmann Publishers, San~Mateo, CA.

\bibitem[\protect\BCAY{Quinlan}{Quinlan}{1993b}]{quinlan/93a}
Quinlan, J.~R. \BBOP1993b\BBCP.
\newblock \BBOQ Combining instance-based and model-based learning\BBCQ\
\newblock In {\Bem Proceedings of the Tenth International Conference on Machine
  Learning}, \BPGS\ 236--243\ University of Massachusetts, Amherst. Morgan
  Kaufmann.

\bibitem[\protect\BCAY{Roth}{Roth}{1970}]{roth/70}
Roth, R.~H. \BBOP1970\BBCP.
\newblock \BBOQ An approach to solving linear discrete optimization
  problems\BBCQ\
\newblock {\Bem Journal of the {ACM}}, {\Bem 17\/}(2), 303--313.

\bibitem[\protect\BCAY{Safavin \BBA\ Landgrebe}{Safavin \BBA\
  Landgrebe}{1991}]{safavin/landgrebe/91}
Safavin, S.~R.\BBACOMMA\  \BBA\ Landgrebe, D. \BBOP1991\BBCP.
\newblock \BBOQ A survey of decision tree classifier methodology\BBCQ\
\newblock {\Bem IEEE Transactions on Systems, Man and Cybernetics}, {\Bem
  21\/}(3), 660--674.

\bibitem[\protect\BCAY{Sahami}{Sahami}{1993}]{sahami/93}
Sahami, M. \BBOP1993\BBCP.
\newblock \BBOQ Learning non-linearly separable boolean functions with linear
  threshold unit trees and madaline-style networks\BBCQ\
\newblock In {\Bem Proceedings of the Eleventh National Conference on
  Artificial Intelligence}, \BPGS\ 335--341. AAAI Press.

\bibitem[\protect\BCAY{Salzberg}{Salzberg}{1991}]{salzberg/91}
Salzberg, S. \BBOP1991\BBCP.
\newblock \BBOQ A nearest hyperrectangle learning method\BBCQ\
\newblock {\Bem Machine Learning}, {\Bem 6}, 251--276.

\bibitem[\protect\BCAY{Salzberg}{Salzberg}{1992}]{salzberg/92}
Salzberg, S. \BBOP1992\BBCP.
\newblock \BBOQ Combining learning and search to create good classifiers\BBCQ\
\newblock \BTR\ JHU-92/12, Johns Hopkins University, Baltimore MD.

\bibitem[\protect\BCAY{Salzberg, Chandar, Ford, Murthy, \BBA\ White}{Salzberg
  et~al.}{1994}]{salzberg/etal/94}
Salzberg, S., Chandar, R., Ford, H., Murthy, S.~K., \BBA\ White, R.
  \BBOP1994\BBCP.
\newblock \BBOQ Decision trees for automated identification of cosmic rays in
  Hubble Space Telescope images\BBCQ\
\newblock {\Bem Publications of the Astronomical Society of the Pacific}, {\Bem
  to appear}.

\bibitem[\protect\BCAY{Schaffer}{Schaffer}{1993}]{schaffer/93}
Schaffer, C. \BBOP1993\BBCP.
\newblock \BBOQ Overfitting avoidance as bias\BBCQ\
\newblock {\Bem Machine Learning}, {\Bem 10}, 153--178.

\bibitem[\protect\BCAY{Schlimmer}{Schlimmer}{1993}]{schlimmer/93}
Schlimmer, J. \BBOP1993\BBCP.
\newblock \BBOQ Efficiently inducing determinations: A complete and systematic
  search algorithm that uses optimal pruning\BBCQ\
\newblock In {\Bem Proceedings of the Tenth International Conference on Machine
  Learning}, \BPGS\ 284--290. Morgan Kaufmann.

\bibitem[\protect\BCAY{Smith, Everhart, Dickson, Knowler, \BBA\ Johannes}{Smith
  et~al.}{1988}]{smith/etal/88}
Smith, J., Everhart, J., Dickson, W., Knowler, W., \BBA\ Johannes, R.
  \BBOP1988\BBCP.
\newblock \BBOQ Using the {ADAP} learning algorithm to forecast the onset of
  diabetes mellitus\BBCQ\
\newblock In {\Bem Proceedings of the Symposium on Computer Applications and
  Medical Care}, \BPGS\ 261--265. IEEE Computer Society Press.

\bibitem[\protect\BCAY{Utgoff}{Utgoff}{1989}]{utgoff/89b}
Utgoff, P.~E. \BBOP1989\BBCP.
\newblock \BBOQ Perceptron trees: A case study in hybrid concept
  representations\BBCQ\
\newblock {\Bem Connection Science}, {\Bem 1\/}(4), 377--391.

\bibitem[\protect\BCAY{Utgoff \BBA\ Brodley}{Utgoff \BBA\
  Brodley}{1990}]{utgoff/brodley/90}
Utgoff, P.~E.\BBACOMMA\  \BBA\ Brodley, C.~E. \BBOP1990\BBCP.
\newblock \BBOQ An incremental method for finding multivariate splits for
  decision trees\BBCQ\
\newblock In {\Bem Proceedings of the Seventh International Conference on
  Machine Learning}, \BPGS\ 58--65. Los Altos, CA. Morgan Kaufmann.

\bibitem[\protect\BCAY{Utgoff \BBA\ Brodley}{Utgoff \BBA\
  Brodley}{1991}]{utgoff/brodley/91}
Utgoff, P.~E.\BBACOMMA\  \BBA\ Brodley, C.~E. \BBOP1991\BBCP.
\newblock \BBOQ Linear machine decision trees\BBCQ\
\newblock \BTR~10, University of Massachusetts at Amherst.

\bibitem[\protect\BCAY{Van~de Merckt}{Van~de Merckt}{1992}]{deMerckt/92}
Van~de Merckt, T. \BBOP1992\BBCP.
\newblock \BBOQ {NFDT}: A system that learns flexible concepts based on
  decision trees for numerical attributes\BBCQ\
\newblock In {\Bem Proceedings of the Ninth International Workshop on Machine
  Learning}, \BPGS\ 322--331.

\bibitem[\protect\BCAY{Van~de Merckt}{Van~de Merckt}{1993}]{deMerckt/93}
Van~de Merckt, T. \BBOP1993\BBCP.
\newblock \BBOQ Decision trees in numerical attribute spaces\BBCQ\
\newblock In {\Bem Proceedings of the 13th International Joint Conference on
  Artificial Intelligence}, \BPGS\ 1016--1021.

\bibitem[\protect\BCAY{Weiss \BBA\ Kapouleas}{Weiss \BBA\
  Kapouleas}{1989}]{weiss/kapouleas/89}
Weiss, S.\BBACOMMA\  \BBA\ Kapouleas, I. \BBOP1989\BBCP.
\newblock \BBOQ An empirical comparison of pattern recognition, neural nets,
  and machine learning classification methods\BBCQ\
\newblock In {\Bem Proceedings of the 11th International Joint Conference of
  Artificial Intelligence}, \BPGS\ 781--787. Detroit, MI. Morgan Kaufmann.

\bibitem[\protect\BCAY{Wolpert}{Wolpert}{1992}]{wolpert/92}
Wolpert, D. \BBOP1992\BBCP.
\newblock \BBOQ On overfitting avoidance as bias\BBCQ\
\newblock \BTR\ SFI TR 92-03-5001, The Santa Fe Institute, Santa
Fe, New Mexico.

\end{thebibliography}


\end{document}


